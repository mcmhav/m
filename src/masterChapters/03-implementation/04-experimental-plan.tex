% !TEX root = ../../report.tex

\section{Experimental Plan}

%We can sometimes evaluate how well the recommender achieves its overall goals.
%For example, we can check an e-commerce website revenue with and without the
%recommender system and thereby estimate the value of the system to the website.

Section \ref{sec:evaluation} covers a wide range of evaluation metrics that
measure different properties of the recommender system. This section will cover
our experimental plan, starting off by looking at the goals for our experiments.
The remaining parts of the section will describe the datasets used for evaluation,
our evaluation methodology and our evaluation metrics of choice.
The section will also describe the datasets used for evaluation, our evaluation methodology

We have the following goals for our experiment:

\begin{itemize}
	\item Does our proposed implicit rating methods improve the recommendation quality over
	binary preference data?
	\item Compare the different implicit rating functions.
	\item Select the best combination of methods for the SoBazar recommender system. 
\end{itemize}

\marginpar{Any suggestions?}

Our first goal is to see whether or not our implicit rating method actually improve the
recommendation quality over using binary preference data.

We also wish to compare the different implicit rating mapping functions.
\marginpar{TODO: Discussion on how}

where the goal of the experiment is to evaluate the properties
of the system, which we have identified as the most important for the systems success,
and select the method that performs the best overall with respect to these properties.

\subsection{Selecting datasets for evaluation}

In addition to evaluate the methods on the Sobazar dataset we want to make sure that our
solution generalizes beyond our experimental dataset, in accordance to the general guidelines
for experimental studies \cite{Shani2011}. The data used for offline evaluation should match
as closely as possible the data we expect the recommender system to face when it is
deployed \cite{Gunawardana2009}. When selecting datasets for evaluation we focused on the
following dataset properties:

\begin{itemize}
	\item Size of dataset: Preferable as close as possible to Sobazar (6 months from now)
	in terms of number of ratings, users and items.
	
	\item Different types of user feedback: Preferable different types of implicit feedback
	such as browsing and buying behavior.
	
	\item Domain: Preferably a domain as close to possible as the e-commerce domain with respect
	to the importance of factors such as recentness.
	
	\item Presence of features: To evaluate the hybrid methods.
	
	\item Timestamps: To evaluate the recentness mapping
\end{itemize}

We were unable to acquire any e-commerce datasets containing user browsing history, purchases etc.
And we therefore had to turn for other domains for datasets...

%The first dataset we chose for evaluation was the MovieLens 1M dataset. The reason for selecting the MovieLens 1M dataset over MovieLens 100K is that the number of users is closer to what we expect the SoBazar to have after the \emph{official} launch this summer in addition to having user- and item features. The MovieLens dataset can also be seen as a \emph{benchmark} dataset as it is one of the most popular recommender systems dataset used for evaluation in countless articles.

\subsubsection{Book-Crossing Dataset?}

Use implicit feedback as \emph{item-clicked} and explicit ratings greater than 5 as \emph{likes}.


%General statistics and averages
%Interesting findings/properties
%Was cleaning neccesary?
%How was the methods evaluated on the dataset?
%	- x-fold cross validation


%\subsubsection{MovieLens 1M}
%
%As our second dataset we have chosen the MovieLens 1M dataset. The dataset contain 1,000,209 anonymous ratings on approximately 3,706 movies (The readme mentions 3900 movies) made by 6040 users. User features included age, gender, occupation and zipcode, item features include movie genre. Each user included in the dataset have provided a minimum of 20 ratings. The average rating given to movies in the dataset is $3.58$. There are 18 different genres in the dataset, but each movie can have multiple genres assigned to it, of which there are 498 different combinations in the dataset.
%
%\begin{table}[H]
%\centering
%\begin{tabular}{|l|l|}
%\hline
%Male & Female \\ \hline
%4331 & 1709 \\ \hline
%\end{tabular}
%\caption{MovieLens Gender Distribution}
%\end{table}
%
%\begin{table}[H]
%\centering
%\begin{tabular}{|l|l|l|l|l|l|l|}
%\hline
%Under 18 & 18-24 & 25-34 	& 35-44 	& 45-49 & 50-55 & 56+ \\ \hline
%222		 &	1103 &	2096	&	1193	& 550	& 496	& 380 \\ \hline
%\end{tabular}
%\caption{MovieLens 1M Age Group Distribution}
%\end{table}
%
%\begin{table}[H]
%\centering
%\begin{tabular}{|l|l|}
%\hline
%Other/not specified  & 711  \\ \hline
%Academic/Educator  & 528  \\ \hline
%Artist  & 267 \\ \hline
%Clerical/Admin & 173 \\ \hline
%College/Grad student  & 759 \\ \hline
%Customer service & 112 \\ \hline
%Doctor/Health care & 236 \\ \hline
%Executive/Managerial & 679 \\ \hline
%Farmer & 17 \\ \hline
%Homemaker & 92 \\ \hline
%K-12 student & 195 \\ \hline
%Lawyer & 129 \\ \hline
%Programmer & 388 \\ \hline
%Retired & 142 \\ \hline
%Sales/Marketing & 302 \\ \hline
%Scientist & 144 \\ \hline
%Self-employed & 241 \\ \hline
%Technician/Engineer & 502 \\ \hline
%Tradesman/Craftsman & 70 \\ \hline
%Unemployed & 72 \\ \hline
%Writer & 281 \\ \hline
%\end{tabular}
%\caption{MovieLens 1M Occupation Distribution}
%\end{table}
%
%As the sparsity of this dataset is fairly low (95.53164), we decided to evaluate this dataset using the holdout method. As the dataset include time-stamps we split the dataset based on time, meaning that all ratings given before a given time will be used to train the model and all ratings given after this point will be used as a testset. We set aside the last 25\% ratings for evaluation and train the model using the remaining 75\%.

\subsubsection{The Sobazar Dataset}

%Highlight limitations of the sobazar dataset for evaluation.
	%-> Implications for evaluation
The Sobazar dataset is smallest and sparsest of our datasets used for evaluation.
The dataset contain ratings 15,252 given by 1,235 users to 3,386 items.
We also have access to semi-structured product information collected/crawled from
the online retailers for \emph{some} items. In addition user data from the users
can also be downloaded from Facebook.

Having such a small and sparse dataset has several implications. Firstly we have
to avoid \emph{wishful thinking} as we have very thin data, meaning that we cannot
rely on getting reliable results. Secondly, our evaluation methodology must be
\emph{tailored} for small sparse datasets. When using cross-validation the number
of folds depends on the size of the dataset. For large datasets, even 3-fold Cross
Validation will be quite accurate, while for very sparse datasets, we may have to
use leave-one-out in order to train on as many examples as possible. The advantages
of using a large number of folds is that the bias of the true error rate estimators
will be small (the estimator will be very accurate), with the disadvantages being that
the variance of the true error rate estimator will be large in addition to increased
computation time. To exemplify this we ran a small experiment on the Sobazar data using
IBCF, with $k-NN=3$, experimenting with different $K$-fold values:

\begin{table}[H]
\centering
\begin{tabular}{|l|l|l|l|l|}
\hline
K-fold & 	$min_{RMSE}$ 	&	$max_{RMSE}$ 	& Average 	& Variance 					\\ \hline
3	   & 	0.783 			& 	0.789 			& 0.785 	& $8.730 \times 10^{-6}$	\\ \hline
5	   & 	0.759			& 	0.802 			& 0.781 	& $3.363 \times 10^{-4}$ 	\\ \hline
8	   & 	0.740			& 	0.781			& 0.758 	& $1.656 \times 10^{-4}$ 	\\ \hline
10	   & 	0.718 			& 	1.026			& 0.810  	& 0.0125					\\ \hline
\end{tabular}
\caption{Evaluation results from experimenting with different k-fold splits on the Sobazar dataset}
\end{table}

\marginpar{This table is not really necessary to include...}

When increasing the number of folds we could see that it was unable to generate
any recommendations at all for some folds, or getting really poor results, meaning
that we get some difficult splits. E.g. when using 30 folds, IBCF was unable to
provide any recommendations for 8 folds out of 30. By using more than 10 folds
it is increasingly likely that we end up with a few or more \emph{unrecommendable} 
instances in the test set, yielding no test result. Based on these results,
and the general consensus that more folds are better for small datasets we
believe that using between $5-10$-folds would be a good choice for model validation.
Another alternative well suited for sparse datasets is the \emph{all but one} or
\emph{leave one out} method, in which one remove one rating from the test users
and try to predict the hidden ratings.

Another important concern is whether or not to take the timestamps into consideration,
which directly speaks against the use of cross-validation, as we wish to use the past
interactions to predict future actions. When using the \emph{leave one out} method one
could e.g. select a predetermined set of test users based on some criteria and remove
their latest rating and try to predict it and repeat the process any number of times.

%TODO - Cross-validation time-based splits?
%TODO - Look at what others have done

\subsubsection{Overview of the Datasets}

Table \ref{table:datasets} shows an overview of the datasets used for evaluation.

%TODO - What else is interesting to know? Rating scale, average number of ratings per user, number of cold start users...
%TODO - % of users with less than 5 ratings for both datasets

\begin{table}[H]
    \centering
    \begin{tabular}{|l|l|l|l|l|l|}
    \hline
	Dataset			& 	Ratings 	& 	Users	& 	Items 	& 	Sparsity	& Rating Scale 				\\ \hline
	Sobazar 		& 	15,252  	& 	1,235	&	3,386	&	99.65599	& Implicit Ratings(1-5)		\\ \hline
	%Movielens 1M	& 	1,000,029   &	6040 	&	3706	&	95.53164	& Explicit (1-5)			\\ \hline
	Dataset 2 		& 	-  			& 	-		&	-		&	-			&							\\ \hline
    \end{tabular}
    \label{table:datasets}
    \caption [Overview of the datasets used for evaluation]{Overview of the datasets used for evaluation}
\end{table}


\subsection{Simulating the Cold-Start Problem}

To simulate the cold-start problem and evaluate how well our the different
methods tackle the different cold-start situations we came up with the following
evaluation methodology. As mentioned in Section \ref{sec:cold-start-eval} there is
no common framework for assessing the cold-start performance of recommender systems.
Our goal is to come up with \emph{comprehensive} framework to assess the cold-start
performance of our recommender systems. The following inputs changes the dataset over time:

\begin{itemize}
	\item 	Existing users watch new items in the catalogue
	\item	New users join the system and view their first item
	\item	New items are added to the catalogue
\end{itemize}

The first input source has the effect of increasing the dataset density, the average user
profile length, and the average number of views per item. The second input factor has
the effect of decreasing both the dataset density and the average user profile length,
as the new users that join the system have watched one movie. Similarly, the third input
factor has the effect of decreasing both the dataset density and the average number of
views per item.

To simulate the cold-start user problem we propose splitting the users into two disjoint
sets, similarly as in \cite{Stern2009, Lam2008}, using 90\% of the users for training and
setting aside the remaining 10\% for evaluation. We then train the model with e.g. 5, 15,
25 and 35 ratings and predict the remaining values. This process is repeated $k$ times.

Similarly, to simulate the cold-start item problem we again split the items into two
disjoint sets, using 90\% of the items for training and the remaining 10\% for
evaluation.  We then train the model with e.g. 20, 40, 60 and 80 ratings and predict
the remaining values. This process is repeated $k$ times.

The selection criteria for test items and users can differ from dataset from dataset.
E.g. in \cite{Rashid2002, Rashid2008} the authors selected a subset of the users with
more than 200 ratings, but you can not expect 10\% all the users for all datasets
to have provided 200 ratings, so this number might be lowered if necessary. The 
implications of removing the top 10\% of the raters from the Sobazar dataset is 
fairly large as they stand for a large portion of the few ratings we have.

To evaluate the cold-start system performance we use the same method as described
in ~\cite{Agarwal2009} where the authors propose using a 75:25 training/test split,
where we at random draw e.g. 35\%, 50\% and then finally use all (75\%) of the
ratings in the training set and predict the remaining 25\%. This process is
repeated $k$ times.

%TODO - How is this implemented on the sobazar data?
For the Sobazar dataset we select 10\% of the users as test users, for a user to be
selected as a test user, the user must have provided at least 25 ratings. We then train
the model using 10, 15, 20 of their ratings and try to predict their remaining ratings.
As we have a very low number of ratings and a large item collection we had to use
only 2\% of the items as test items, where each test item have been rated atleast 20 times.
We train the model using 5, 10, 15 of their ratings and try to predict their remaining
ratings. To evaluate the cold-start system performance we split the dataset in a test
and training set using 20\% of the ratings for testing and then train the model
using 40\%, 60\% and 80\% of the ratings for training.


%TODO - How is this implemented on the x dataset?


\subsection{Evaluation Metrics}

%TODO - Discussion of evaluation metrics

%http://www.slideshare.net/gunnar-schroeder	
	
A large variety of metrics have been published, and some of these metrics are highly correlated \cite{Herlocker2004}.

Little guidance for evaluating recommender systems and choosing metrics.

Some important questions to ask oneself is:

\begin{itemize}
	\item Which aspects of the usage scenario and the data influence the choice?
	\item Which metrics are applicable?
	\item What does these metrics express?
	\item What are the differences among them?
	\item Which metric represent our user-case best?
	\item How much do the metrics suffer biases?
\end{itemize}

What factors influence the choice of evaluation metrics?

Business goals, User interests
Recommender task and interaction (Prediction, Classification, Ranking...)
Preference data (Explicit, Implicit, ...)

Area under curve (AUC)

%Why AUC?
ROC related measures are highly popular \cite{Agarwal2009, Gantner2010, Schein2002, Liu2011, Gantner2010}

AUC vs. Limited Area under curve|

Useful metric for top-k recommender evaluations

MAP@K

%TODO - Justify why these metrics combined will give a good overview of the performance.


%TODO - Cold-start evaluation Metrics

In addition to looking at the above mentioned metrics it would be interesting to see how
the different sparsity levels affect both the user- and item-space coverage of the different
when evaluating the cold-start system performance. The user-space coverage is the number of
users the recommender is able to produce recommendations for while the item-space
coverage is measured by looking at how many of the items are recommendable.



