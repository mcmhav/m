% !TEX root = ../report.tex

\chapter{Introduction}
\minitoc
\setcounter{page}{1}
\pagenumbering{arabic}

\clearpage

\section{Motivation}

%TODO (HELGE) Flere cites, jobbe med flyten i språket, konkretisere}
%TODO Finish introduction

%What are recommender systems good for?
In today's day and age the increasing amount of data overwhelm our human processing capabilities in many information seeking tasks. To cope with this overload researchers have introduced recommender system to filter the ever increasing information and only present a small selection of items which reflects the users tastes, interests and priorities. Recommender systems are an active research field and has been successfully applied to many different services ranging from e-commerce sites such as \emph{Amazon}, movie and TV-series streaming services like \emph{Netflix} and in different music applications such as \emph{Last.fm} and \emph{iTunes}.

%What is telenors incentive?
Many of the largest commerce Web sites have been using recommender systems to help their customers find products to purchase for nearly two decades.
Schafer et. al. \cite{Schafer1999} identified three ways, in which recommender systems increase E-commerce sales: (1) Browser into buyers: Recommender systems can help customers find products they wish to purchase, (2) Cross-sell: Recommender systems improve cross-sell by suggesting additional products for the customer to purchase and (3) Loyalty: In a world where the competitor only is one click away, gaining customer loyalty is an essential customer strategy. Recommender systems improve loyalty by creating a value added relationship between the site and the customer.

SoBazar is a new fashion e-commerce application for web and hand held devices developed by Telenor, Norway's largest Telco company. The application aggregates fashion products from various brands and stores into one \emph{webstore}. The app is planned to be officially \emph{launched} this summer, meaning that there currently is a limited amount user-item interaction data available, making it a classic cold-start scenario. We have access to data coming from multiple sources including user information from Facebook, rich meta-data description of the items from the retailers as well as information about the users browsing and buying habits collected by the application.

In a classic recommendation scenario one has access to rich explicit information regarding the users preferences, often in the form of ratings. We have a very limited amount of user-item interaction information available and must therefore determine the best way to leverage the implicit information collected by the application in combination with the user combination collected from facebook and item meta-data to improve the recommendations.

The fashion recommendation task also differs from movies, books and music in many ways.
Firstly, fashion consumption is largely determined by seasons.
E.g. one does not buy winter jackets in the middle of summer.
Some clothes do also go out of fashion, the same can not be said for \emph{all} movies.
There are a whole different set of important aspects regarding the items or products when recommending in the fashion domain, such as: brand, color and size.

For the average consumer of a movie the producer might not affect greatly the way the consumer views the movie, but when it comes to fashion the consumer might mainly look at the brand of the product when deciding what to consume.
The social aspect also affects fashion recommendation on another level than movies.
The affiliation of a member of a social group might not be much affected by what movies the member likes or dislikes, but what the individual wears can greatly affect it~\cite{vignali2009fashion}.
How a consumer consumes differs from the named domains, and then again how to recommend will differ.
Fashion is often used to show of to fellow peers, and will often produce satisfaction for the individual showing of.
Whereas the satisfaction of a movie or book can be just as great in solitude, rather than in the company of others.
This magnifies the importance of the users or consumers social group.
This is where facebook and other social networks can be of great help when recommending fashion products.

%The user are logging into the application through their Facebook accounts.
%This opens for the potential to explore the trust based recommendation domain.
%In the application the user will have the option as: to watch, to like and to buy products, from the various brands.
%All these events are stored to be potentially used for analysis for different purposes, such as product recommendation.
%The events, mentioned above, will mainly be implicit feedback.
%With this implicit feedback recommendations can be done, and user experience can be improved.
%As mentioned, there are a lot of research done on recommender systems, regarding systems like Netflix and Amazon.
%However research done on the combination of implicit feedback and the fashion domain in comparison to explicit feedback and domains such as the movie domain is small, which makes this an cutting edge topic to explore in depth.
%The main task of this thesis is therefore to provide the SoBazar application with recommendation based on the implicit feedback.

\section{Problem Statement}

\subsection{Implicit feedback}
\begin{enumerate}
  \item How do we generate and consequently evaluate implicit ratings generated by implicit feedback.
  \item How do we evaluate our generated implicit ratings.
  \item Establish user interaction patterns to better evaluate.
\end{enumerate}

\subsection{Fashion domain}
\begin{enumerate}
  \item What are the domain specific challenges? (E-commerce, fashion)
\end{enumerate}

\subsection{Cold start}
\begin{enumerate}
  \item How to best solve the cold-start problem (Given our current limitations)
\end{enumerate}

% \todo{up for discussion/more to come}

\subsection{Other Challenges}
Pragmatic approach - start off simple, extend system
Scalability / extensibility issues (Ensure that the system can handle the increased load as it goes live)

% #1 - How to make content-based recommendations based on content information from multiple providers in (two)different languages
%   Goal: Select features/feature engineering based on literature study + observations. What features are the most informative? #2
% #4 - How to weight a blending of recommendation algorithms as time progresses.
%   Goal: Use weights based on evaluation and findings through literary study

\section{System Overview}
  \marginpar{maybe reference to the chapters/section the boxes are referring to}
  \begin{center}
    \begin{tikzpicture}
      [node distance = 1cm, auto,font=\footnotesize,
      % STYLES
      every node/.style={node distance=1.5cm},
      % The comment style is used to describe the characteristics of each process
      comment/.style={rectangle, inner sep= 5pt, text width=4cm, node distance=0.25cm, font=\scriptsize\sffamily},
      % The nonProcess style
      nonProcess/.style={rectangle, draw, inner sep=5pt, text width=4cm, text badly centered, minimum height=1.2cm, font=\footnotesize\sffamily},
      % The process style is used to draw the processs' name
      process/.style={rectangle, draw, fill=black!10, inner sep=5pt, text width=4cm, text badly centered, minimum height=1.2cm, font=\bfseries\footnotesize\sffamily}]

      % Draw processs
      \node [nonProcess] (inputData) {Input data based on implicit feedback (events)};
      \node [process, below of=inputData] (implicitConverter) {Convert implicit feedback to implicit ratings};
      \node [nonProcess, below of=implicitConverter] (ratings) {Ratings for the items};
      \node [process, below of=ratings] (recommendations) {Make recommendations};
      \node [process, below of=recommendations] (evaluations) {Evaluate recommendations};
      \node [nonProcess, below of=evaluations] (evaluationValue) {Recommendation score};

      %%%%%%%%%%%%%%%
      % Comments
      \node [comment, right=0.25 of inputData] (comment-inputData) {
        Feedback gathered from the soBazar application data dump
      };
      \node [comment, right=0.25 of implicitConverter] (comment-implicitConverter) {
        Coverts the inputed implicit feedback to implicit ratings based on different conversion schemes
      };
      \node [comment, right=0.25 of ratings] (comment-ratings) {
        The output of the conversion is a set of ratings on the different items from the soBazar dataset
      };
      \node [comment, right=0.25 of recommendations] (comment-recommendations) {
        Makes recommendations based on the inputed ratings. Different approaches to make the recommendations can be take, such as matrix factorization or neighborhood based approaches
      };
      \node [comment, right=0.25 of evaluations] (comment-evaluations) {
        Evaluate the resulting predictions from the recommendations. Different evaluation metrics can be used, such as AUR and nDCG
      };
      \node [comment, right=0.25 of evaluationValue] (comment-evaluationValue) {
        The score of the recommendations done on the ratings
      };
      %%%%%%%%%%%%%%%%

      % Draw the links between processs
      \path[->,thick]
        (inputData) edge (implicitConverter)
        (implicitConverter) edge (ratings)
        (ratings) edge (recommendations)
        (recommendations) edge (evaluations)
        (evaluations) edge (evaluationValue);
    \end{tikzpicture}
    \captionof{figure}[System Overview]{Overview of the system. Boxes in white represents input and output data. Boxes in gray represents processes}
  \end{center}

\section{Outline}
\begin{table}[H]
  \centering
  \begin{tabularx}{\textwidth}{ l X l }
    \textbf{Chapter}      & \textbf{Description} \\
    \hline \\ [-1.5ex]
    Chapter 1 & The Introduction chapter gives an overview of the project to the reader. It also outlines the purpose and motivation of the project. \\
    \hline \\ [-1.5ex]
    Chapter 2 & The Preliminary Study chapter documents knowledge, research and technology that is relevant to the project, and how and why some of them were prioritized over others when it comes to how they are used in the project. \\
    \hline \\ [-1.5ex]
    Chapter 3 & The Requirements chapter describes the requirements of the project. It also describes how and why they were created. \\
    \hline \\ [-1.5ex]
    Chapter 4 & The implementation chapter describes the design of the system and how the design has be implemented. \\
    \hline \\ [-1.5ex]
    Chapter 5 & Evaluation chapter discussed the development process, testing of results and major issues. \\
    \hline \\ [-1.5ex]
    Chapter 6 & The Conclusion chapter sums up the project and describes the findings and reflects on them. It also describes further work to be done. \\
    \hline \\ [-1.5ex]
    Appendix & The appendix contains extended information such as a full list of the requirements. \\
  \end{tabularx}
  \caption{Structure and chapters of the report.}
  \label{table-reportstructure}
\end{table}
