% !TEX root = ../report.tex

\chapter{Introduction}
\minitoc
\setcounter{page}{1}
\pagenumbering{arabic}
\clearpage

\section{Motivation}
\label{sec:motivation}

%TODO (HELGE) Flere cites, jobbe med flyten i språket, konkretisere}
%TODO Finish introduction

%What are recommender systems good for?
In today's day and age the increasing amount of data overwhelm our human
processing capabilities in many information seeking tasks. To cope with this
overload researchers have introduced recommender system to filter the ever
increasing information and only present a small selection of items which
reflects the users tastes, interests and priorities. Recommender systems are an
active research field and has been successfully applied to many different
services ranging from e-commerce sites such as \emph{Amazon}, movie and
TV-series streaming services like \emph{Netflix} and in different music
applications such as \emph{Last.fm} and \emph{iTunes}.

%What is telenors incentive?
Many of the largest commerce Web sites have been using recommender systems to
help their customers find products to purchase for nearly two decades.
Schafer et. al. \cite{Schafer1999} identified three ways, in which recommender
systems increase E-commerce sales: (1) Browser into buyers: Recommender systems
can help customers find products they wish to purchase, (2) Cross-sell:
Recommender systems improve cross-sell by suggesting additional products for
the customer to purchase and (3) Loyalty: In a world where the competitor only
is one click away, gaining customer loyalty is an essential customer strategy.
Recommender systems improve loyalty by creating a value added relationship
between the site and the customer.

\marginpar{TODO: We DO NOT have facebook data}
SoBazaar is a new fashion e-commerce application for web and hand held devices
developed by Telenor, Norway's largest Telco company. The application
aggregates fashion products from various brands and stores into one
\emph{webstore}. The app is planned to be \emph{officially launched} this
summer, meaning that there currently is a limited amount user-item interaction
data available, making it a classic cold-start scenario. We have access to data
coming from multiple sources including user information from Facebook, rich
meta-data description of the items from the retailers as well as information
about the users browsing and buying habits collected by the application.

In a classic recommendation scenario one has access to rich explicit
information regarding the users preferences, often in the form of ratings. We
have a very limited amount of user-item interaction information available and
must therefore determine the best way to leverage the implicit information
collected by the application in combination with the user combination collected
from Facebook and item meta-data to improve the recommendations.

This scenario differs from making recommendations on movies, books and music, not just in the lack of rich explicit ratings, but also in the context of the domain the recommendations will be done, namely the fashion domain.
Firstly, fashion consumption is largely determined by seasons.
E.g. one does not buy winter jackets in the middle of summer.  Some clothes do
also go out of fashion, the same can not be said for \emph{all} movies. Secondly, there are a whole different set of important aspects regarding the items or products
when recommending in the fashion domain, such as: brand, color and size.

\marginpar{TODO: Provide a useful and motivating example, to better understand
problem}

For the average consumer of a movie the producer might not affect greatly the
way the consumer views the movie, but when it comes to fashion the consumer
might mainly look at the brand of the product when deciding what to consume.
The social aspect also affects fashion recommendation on another level than
movies.
The affiliation of a member of a social group might not be much affected by
what movies the member likes or dislikes, but what the individual wears can
greatly affect it~\cite{vignali2009fashion}.
How a consumer consumes differs from the named domains, and then again how to
recommend will differ.  Fashion is often used to show off to fellow peers, and
will often produce satisfaction for the individual showing of.
Whereas the satisfaction of a movie or book can be just as great in solitude,
rather than in the company of others.  This magnifies the importance of the
users or consumers social group.  This is where Facebook and other social
networks can be of great help when recommending fashion products.

%The user are logging into the application through their Facebook accounts.
%This opens for the potential to explore the trust based recommendation domain.
%In the application the user will have the option as: to watch, to like and to
%buy products, from the various brands.  All these events are stored to be
%potentially used for analysis for different purposes, such as product
%recommendation.  The events, mentioned above, will mainly be implicit
%feedback.  With this implicit feedback recommendations can be done, and user
%experience can be improved.  As mentioned, there are a lot of research done on
%recommender systems, regarding systems like Netflix and Amazon.  However
%research done on the combination of implicit feedback and the fashion domain
%in comparison to explicit feedback and domains such as the movie domain is
%small, which makes this an cutting edge topic to explore in depth.  The main
%task of this thesis is therefore to provide the SoBazaar application with
%recommendation based on the implicit feedback.

\section{Problem Statement and Goals}

\marginpar{TODO: Overall problem statement}
%Problem statement
The primary aim of this thesis is to propose a design for a recommender system, that may be used in an
e-commerce fashion application. The designed system should be able to generate recommendations based on
user-interactions with the applications, and optionally use product information to improve the recommendation
quality. In addition, the design should provide a solution to the problem of giving accurate recommendations
to new users of the system.

Recommender systems have been applied to a wide range of different domains including music, movies, e-commerce, news
and many others. Since each application domain has its own specific needs, the methods used for recommendations differ.
This leads us to our first research goals, which are the following:

\begin{itemize}
	\item G1: Gain a better understanding of the fashion domain.
  	\item G2: Identify the specific challenges of making fashion recommendations.
  	\item G3: How can existing technologies be adapted to mitigate or overcome these challenges?
\end{itemize}

Most recommender systems base their recommendations of previous feedback given by the user. A central problem for
recommender systems is therefore the cold-start problem. How do you recommend items when you have little or no
user feedback. SoBazaar is a \emph{brand new} application, and have therefore naturally recorded a limited amount
of user-interactions. Poor recommendations can result in customer defection and loss of revenue for Telenor.
The above reasoning lead us to the following goal, which is to:

\begin{itemize}
  \item G4: Find the existing solutions to the cold-start problem presented in the recommender system literature
  		and present the possible solution(s) best suited for our application and needs.
\end{itemize}

We focus mainly on finding \emph{complete} solutions to the cold-start problem, that can handle both new users, new items
and general sparsity related problems.

As previously mentioned, recommender systems base their recommendations on feedback given by the user. SoBazaar records
all the users interactions with the application, which then could be interpreted as user feedback given by users to items.
We have multiple types of interactions such as e.g. browsing, wanting and purchasing items. We would like to figure out
how these events could be used to learn the users true preferences. This lead us to the following goals:

\marginpar{TODO: Assumptions and Constraints}
%E.g. Due to data limitations we have excluded cold-start articles looking at how demographic data can be incorporated to solve the cold-start user problem.

\begin{itemize}
 	\item G5: Explore the existing solutions of how to infer user preference from implicit feedback data.
 	\item G6: Establish user interaction patterns to support our assumptions.
	\item G7: Find different methods of combining various event types into \emph{implicit ratings}?
  	\item G8: Evaluate the \emph{implicit ratings}.
\end{itemize}

We would like to combine the solutions found through our work with the above mentioned goals and combine these in a
proposal to a design for a fashion e-commerce recommender system.

\section{System Overview}

  \marginpar{Include filterbots in a more detailed system overview?}
  
  This thesis proposes a system for making recommendations based on implicit
  data, where the input is implicit feedback collected on a set of users and
  items and the output is personalized recommendations to all users, proposing
  new and relevant products to their preferences.
 
  \begin{figure}[H]
    \centering
    \begin{tikzpicture}
      [node distance = 1cm, auto,font=\footnotesize,
      % STYLES
      every node/.style={node distance=1.5cm},
      % The comment style is used to describe the characteristics of each process
      comment/.style={rectangle, inner sep= 5pt, text width=4cm, node distance=0.25cm, font=\scriptsize\sffamily},
      % The nonProcess style
      nonProcess/.style={rectangle, draw, inner sep=5pt, text width=4cm, text badly centered, minimum height=1.2cm, font=\footnotesize\sffamily},
      % The process style is used to draw the processs' name
      process/.style={rectangle, draw, fill=black!10, inner sep=5pt, text width=4cm, text badly centered, minimum height=1.2cm, font=\bfseries\footnotesize\sffamily}]

      % Draw processs
      \node [nonProcess] (inputData) {Input data based on implicit feedback (events)};
      \node [process, below of=inputData] (implicitConverter) {Convert implicit feedback to implicit ratings};
      \node [nonProcess, below of=implicitConverter] (ratings) {Ratings};
      \node [process, below of=ratings] (recommendations) {Recommending};

      \node [comment, right=0.25 of inputData] (comment-inputData) {
        Implicit Feedback (clicks, purhcases etc.) collected from the SoBazaar
        analytics logs. Detailed description in Section~\ref{sec:sobazaar-data}
      };

      \node [comment, right=0.25 of implicitConverter] (comment-implicitConverter) {
        Converts the inputed implicit feedback to implicit ratings based on
        different conversion schemes. Detailed description in
        Section~\ref{sec:implementation-implicit}
      };

      \node [comment, right=0.25 of ratings] (comment-ratings) {
        Set of rating triplets, on the form \textit{user, product, rating}.
      };

      \node [comment, right=0.25 of recommendations] (comment-recommendations) {
        Makes recommendations based on the inputed ratings. Different
        approaches to make the recommendations can be take, such as matrix
        factorization or neighborhood based approaches. Detailed description in
        Section~\ref{sec:making-recommendations}
      };

      % Draw the links between processs
      \path[->,thick]
        (inputData) edge (implicitConverter)
        (implicitConverter) edge (ratings)
        (ratings) edge (recommendations);

    \end{tikzpicture}
    \caption{Overview of the system. Boxes in white represents input and output
    data. Boxes in gray represents processes}
  \end{figure}

  Traditionally in most recommender systems this process does not include the
  first two stages of our system, hence going directly from a set of explicitly
  provided ratings to a set of recommendations. However, as we will see, making
  ourselves non-dependent on explicit ratings can both provide us with a good
  set of recommendations and take us one step closer to true artificial
  intelligence, as we require no extra effort from the users – rather learning
  preferences from their behaviour.

  \paragraph{The limitiations} of this thesis are primarly linked to having
  only one available dataset with both sparse and modest data. As a result,
  making conclusions based user behaviour proved difficult due to lack of
  statistical significance on many observations. Limitiations in both time
  available and number of existing users with sufficient activity made is also
  impossible to conduct a thorough user study, which could confirm possible
  found user patterns. 

  In addition, as seen in the overview, external sources of information such as
  social networks and trust-based networks are not considered – as the dataset
  provided did not warrant this. As the products in the SoBazaar application
  are aggregated from multiple sources, and hence product databases, the
  quality of information are notably diverse. Consequently, since extracting
  basic content features different from product titles became close to
  impossible without large engineering efforts, the main focus does not lie on
  hybrid recommender systems (combining content and user-based
  recommendations), although some work is done in this area.

\section{Outline}
\begin{table}[H]
  \centering
  \begin{tabular}{lp{11cm}}
  \toprule
    \textbf{Chapter}      & \textbf{Description} \\ 
  \midrule

    Chapter 1 & \textbf{The Introduction} chapter gives an overview of the
    project to the reader. It also outlines the purpose and motivation of the
    project.  \\[1.5ex]

    Chapter 2 & \textbf{The Dataset} chapter presents the results from our
    dataset analysis. \\[1.5ex]

    Chapter 3 & \textbf{The Preliminary Study} chapter documents knowledge,
    research and technology that is relevant to the project, and how and why
    some of them were prioritized over others when it comes to how they are
    used in the project. \\[1.5ex]

    Chapter 4 & \textbf{The implementation} chapter describes the design of the
    system and how the design has be implemented. \\[1.5ex]

    Chapter 5 & \textbf{Evaluation} chapter discussed the development process,
    testing of results and major issues. \\[1.5ex]

    Chapter 6 & \textbf{The Conclusion} chapter sums up the project and describes the
    findings and reflects on them. It also describes further work to be done.
    \\[1.5ex]

    Appendix & \textbf{The Appendix} contains extended information such as a
    full list of the requirements. \\

  \bottomrule
  \end{tabular}
  \caption{Overview of structure and chapters in the thesis}
  \label{table-reportstructure}
\end{table}
