% !TEX root = ../report.tex

\chapter{Introduction}
\label{chap:introduction}
\minitoc
\setcounter{page}{1}
\pagenumbering{arabic}
\clearpage

\section{Motivation}
\label{sec:motivation}

% What are recommender systems good for?
In today's day and age the increasing amount of data overwhelm our human
processing capabilities in many information seeking tasks. To cope with this
overload researchers have introduced recommender system to filter the ever
increasing information and only present a small selection of items which
reflects the users tastes, interests and priorities. Recommender systems are an
active research field and has been successfully applied to many different
services ranging from e-commerce sites such as \emph{Amazon}, movie and
TV-series streaming services like \emph{Netflix} and in different music
applications such as \emph{Last.fm} and \emph{iTunes}.

% What is telenors incentive?
Many of the largest commerce Web sites have been using recommender systems to
help their customers find products to purchase for nearly two decades.
Schafer et al. \cite{Schafer1999} identified three ways, in which recommender
systems increase E-commerce sales: (1) Browser into buyers: Recommender systems
can help customers find products they wish to purchase, (2) Cross-sell:
Recommender systems improve cross-sell by suggesting additional products for
the customer to purchase and (3) Loyalty: In a world where the competitor only
is one click away, gaining customer loyalty is an essential customer strategy.
Recommender systems improve loyalty by creating a value added relationship
between the site and the customer.

% What is the app and what makes it unique?
SoBazaar is a new fashion e-commerce application for web and hand held devices
developed by Telenor, Norway's largest Telco company. The application
aggregates fashion products from various brands and stores into one \emph{news
feed} with recommended products, enabling the user to shop clothes and
accessories effortlessly and without the need of registering an account on
multiple web stores. Currently the application makes global recommendations
based on popular products and trends within social networks --- however, the goal
before launching the application this upcoming summer (\the\year) is to improve
these ratings making them personalized and grounded in a larger array of
features. This has as noted the potential of improving sales, activity and user
satisfaction.

% Limited amount of data -> we need to utilize other sources. Implicit!
As the application is not yet officially released there is a limited amount of
user-item interaction from a set of beta users. In addition users does not have
the possibility to explicitly rate items on a numeric scale, a scheme often
employed by machine learning engineers and application developers in order to
understand their users preferences. Combining these two factors makes the
recommendation scenario both interesting and unique, since we need to utilize
the implicit information contained in user behavior such as clicking, wanting
or buying an item. Limited data both in number of users, but also in activity,
makes the system prone to what is called \textit{cold-start issues}, where
making recommendations with limited information is resolved by a variety of
techniques.

% The fashion domain have certain challenges.
This scenario differs from making recommendations on movies, books and music,
not just in the lack of rich explicit ratings, but also in the context of the
domain the recommendations will be done, namely the fashion domain. Firstly,
fashion consumption is largely determined by seasons. E.g. one does not buy
winter jackets in the middle of summer. Most clothes do also go out of
fashion, the same can not be said for \emph{all} movies. Secondly, there are a
whole different set of important aspects regarding the items or products when
recommending in the fashion domain, such as: brand, color and size.

% Brand and price are important!
Finally, users are highly price and brand-aware. This can be exemplified by
looking at an average movie consumer, where typically the director of the movie
does not greatly affect the way the movie is consumed. However, in the fashion
domain the consumer might mainly look at the product brand when deciding what
to consume. Price preferences are also related to this property where some
users prefer making \textit{a good deal} on periodic sales whilst other buys
clothes not only for individual satisfaction, but to show of or make a
statement.

% What are our goals and purpose?
In our system, there are two sources of information available for making
recommendations: an aggregated product database and event logs capturing user
activities. Out first goal is to find a technique for translating the implicit
feedback into preferences. These preferences are then combined with product
features and techniques for mitigating the cold-start problem. Finally the
user-item preferences are utilized in making recommendations where challenges
related to both domain and data sources are considered.

% Example scenario
For the users of SoBazaar the final product should not infer with the already
established user interface. Instead the news feed will evolve from today
showing the same recommendations to all users, to showing \textit{personalized
recommendations} to every user, based on their \textit{behavior} in the
application. For every user-item interaction the user is implicitly changing
his/hers preference-profile and will then, by just using the application, get
novel and more robust recommendations. Further, as personalized recommendations
hopefully will increase both sales and user activity we will, with a larger set
of data, be able to understand more user patterns and consequently gain a
deeper understanding of the domain.

\section{Problem Statement and Goals}

% Overall problem statement
The primary aim of this thesis is proposing a system, producing personalized
fashion recommendations based on user behaviour and product features, when both
resources are extremely limited in both quality and volume.

% Introduce goals related to domain
In order to propose such a system, one need a exhaustive understanding of both
the fashion domain and available data. In addition a study is required looking
at existing solutions, competitors and idientify potential competitive
advantages. Finally a discussion on how to overcome found challenges with
respect to both understanding the implicit feedback and making recommendations
is needed. Hence, our domain specific goals can be defined as:

% Domain-specific goals
\begin{itemize}
	\item G1: Gain a better understanding of the fashion domain.
  \item G2: Identify the specific challenges of making fashion recommendations.
  \item G3: Study how existing technologies can be adapted to mitigate or
  overcome these challenges.
\end{itemize}

% Introduce goals related to cold-start
In most recommender systems one base future recommendations on existing
feedback already given by the user. An important however is what to recommend
when no feedback has been given, which is the case for any new user. The same
issue is apparent when a new item is added to the application, where we need a
strategy for introducing it to users although the product is not connected to
any existing feedback. Finally, the closely related problem of making
recommendations in scenarios where feedback is extremely sparse need to be
addressed. First impressions are important both for retention and customer
satisfaction and consequently, seperate goals were set specifically concerning
cold-start issues.

\begin{itemize}
  \item G4: Study existing solutions to the cold-start.
  \item G5: Identify the best suited methods with regards to both application
  feedback and domain.
\end{itemize}

% Introduce goals related to implicit feedback
There is no explicit way for users to \textit{rate} or give \textit{negative
feedback} to items in SoBazaar. This forces us to look at user behaviour and
more specifically interactions on items such as \textit{clicking},
\textit{wanting} or \textit{purchasing}. We would like to learn how these
events could be used to predict the users true preferences. Using implicit
feedback yields the following goals:

\begin{itemize}
  \item G6: Explore the existing solutions of how to infer user preference from
  implicit feedback data.
  \item G7: Establish user interaction patterns to support our assumptions.
  \item G8: Identify different methods of combining various event types into
  \emph{implicit ratings}.
  \item G9: Find metrics in order to evaluate the \emph{implicit ratings}.
\end{itemize}

% Introduce limitations
\paragraph{Limitiations} Our problem statement and consequent goals establish
the main theme of this thesis, but simultaneously they introduce some
limitiations. Hence, the reader should be familiar with what this thesis
\textit{not is}.

The main limitiation of this thesis are primarly linked to having only one
available dataset with both sparse and modest data. As a result, making
conclusions based user behaviour proved difficult due to lack of statistical
significance on many observations. Limitiations in both time available and
number of existing users with sufficient activity made is also impossible to
conduct a thorough user study, which could confirm possible found user
patterns.

In addition, as seen in the overview, external sources of information such as
social networks and trust-based networks are not considered – as the dataset
provided did not warrant this. As the products in the SoBazaar application are
aggregated from multiple sources, and hence product databases, the quality of
information are notably diverse. Consequently, since extracting basic content
features different from product titles became close to impossible without large
engineering efforts, the \textit{main focus} does not lie on hybrid recommender
systems (combining content and user-based recommendations), although some work
is done in this area.

\section{System Overview}

\marginpar{Include filterbots in a more detailed system overview?}

This thesis proposes a system for making recommendations based on implicit
data, where the input is implicit feedback collected on a set of users and
items and the output is personalized recommendations to all users, proposing
new and relevant products to their preferences.

\begin{figure}[H]
  \centering
  \begin{tikzpicture}
    [node distance = 1cm, auto,font=\footnotesize,
    % STYLES
    every node/.style={node distance=1.5cm},
    % The comment style is used to describe the characteristics of each process
    comment/.style={rectangle, inner sep= 5pt, text width=4cm, node distance=0.25cm, font=\scriptsize\sffamily},
    % The nonProcess style
    nonProcess/.style={rectangle, draw, inner sep=5pt, text width=4cm, text badly centered, minimum height=1.2cm, font=\footnotesize\sffamily},
    % The process style is used to draw the processs' name
    process/.style={rectangle, draw, fill=black!10, inner sep=5pt, text width=4cm, text badly centered, minimum height=1.2cm, font=\bfseries\footnotesize\sffamily}]

    % Draw processs
    \node [nonProcess] (inputData) {Input data based on implicit feedback (events)};
    \node [process, below of=inputData] (implicitConverter) {Convert implicit feedback to implicit ratings};
    \node [nonProcess, below of=implicitConverter] (ratings) {Ratings};
    \node [process, below of=ratings] (recommendations) {Recommending};

    \node [comment, right=0.25 of inputData] (comment-inputData) {
      Implicit Feedback (clicks, purhcases etc.) collected from the SoBazaar
      analytics logs. Detailed description in Section~\ref{sec:sobazaar-data}
    };

    \node [comment, right=0.25 of implicitConverter] (comment-implicitConverter) {
      Converts the inputed implicit feedback to implicit ratings based on
      different conversion schemes. Detailed description in
      Section~\ref{sec:implementation-implicit}
    };

    \node [comment, right=0.25 of ratings] (comment-ratings) {
      Set of rating triplets, on the form \textit{user, product, rating}.
    };

    \node [comment, right=0.25 of recommendations] (comment-recommendations) {
      Makes recommendations based on the inputed ratings. Different
      approaches to make the recommendations can be take, such as matrix
      factorization or neighborhood based approaches. Detailed description in
      Section~\ref{sec:making-recommendations}
    };

    % Draw the links between processs
    \path[->,thick]
      (inputData) edge (implicitConverter)
      (implicitConverter) edge (ratings)
      (ratings) edge (recommendations);

  \end{tikzpicture}
  \caption{Overview of the system. Boxes in white represents input and output
  data. Boxes in gray represents processes}
\end{figure}

Traditionally in most recommender systems this process does not include the
first two stages of our system, hence going directly from a set of explicitly
provided ratings to a set of recommendations. However, as we will see, making
ourselves non-dependent on explicit ratings can both provide us with a good
set of recommendations and take us one step closer to true artificial
intelligence, as we require no extra effort from the users --- rather learning
preferences from their behavior.

\section{Outline}
\begin{table}[H]
  \centering
  \begin{tabular}{lp{11cm}}
  \toprule
    \textbf{Chapter}      & \textbf{Description} \\
  \midrule

    Chapter \ref{chap:introduction} & The~\nameref{chap:introduction} chapter gives an overview of the
    project to the reader. It also outlines the purpose and motivation of the
    project.  \\[1.5ex]

    Chapter \ref{chap:thesobazaardata} & \nameref{chap:thesobazaardata} chapter presents the results from our
    dataset analysis. \\[1.5ex]

    Chapter \ref{chap:SotA} & The \nameref{chap:SotA} chapter documents knowledge,
    research and technology that is relevant to the project, and how and why
    some of them were prioritized over others when it comes to how they are
    used in the project. \\[1.5ex]

    Chapter \ref{chap:algbackground} & \nameref{chap:algbackground} \\[1.5ex]

    Chapter \ref{chap:implementaion} & The \nameref{chap:implementaion} chapter describes the design of the
    system and how the design has be implemented. \\[1.5ex]

    Chapter \ref{chap:resulteval} & the \nameref{chap:resulteval} chapter discussed the development process,
    testing of results and major issues. \\[1.5ex]

    Chapter \ref{chap:conclusion} & The \nameref{chap:conclusion} chapter sums up the project and describes the
    findings and reflects on them. It also describes further work to be done.
    \\[1.5ex]

    Appendix & \textbf{The Appendix} contains extended information such as a full list of the requirements. \\

  \bottomrule
  \end{tabular}
  \caption{Overview of structure and chapters in the thesis}
  \label{table-reportstructure}
\end{table}
