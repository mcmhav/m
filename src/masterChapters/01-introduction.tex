% !TEX root = ../report.tex

\chapter{Introduction}
\label{chap:introduction}
\minitoc
\setcounter{page}{1}
\pagenumbering{arabic}
\clearpage

\section{Motivation}
\label{sec:motivation}

% What are recommender systems good for?
In today's day and age the increasing amount of data overwhelm our human
processing capabilities in many information seeking tasks. To cope with this
overload researchers have introduced recommender system to filter the ever
increasing information and only present a small selection of items which
reflects the users tastes, interests and priorities. Recommender systems are an
active research field and has been successfully applied to many different
services ranging from e-commerce sites such as \emph{Amazon}, movie and
TV-series streaming services like \emph{Netflix} and in different music
applications such as \emph{Last.fm} and \emph{iTunes}.

% What is telenors incentive?
Many of the largest commerce Web sites have been using recommender systems to
help their customers find products to purchase for nearly two decades.
Schafer et. al. \cite{Schafer1999} identified three ways, in which recommender
systems increase E-commerce sales: (1) Browser into buyers: Recommender systems
can help customers find products they wish to purchase, (2) Cross-sell:
Recommender systems improve cross-sell by suggesting additional products for
the customer to purchase and (3) Loyalty: In a world where the competitor only
is one click away, gaining customer loyalty is an essential customer strategy.
Recommender systems improve loyalty by creating a value added relationship
between the site and the customer.

% What is the app and what makes it unique?
SoBazaar is a new fashion e-commerce application for web and hand held devices
developed by Telenor, Norway's largest Telco company. The application
aggregates fashion products from various brands and stores into one \emph{news
feed} with recommended products, enabling the user to shop clothes and
accessories effortslessly and without the need of registering an account on
multiple webstores. Currently the application makes global recommendations
based on popular products and trends within social networks – however, the goal
before launching the application this upcoming summer (\the\year) is to improve
these ratings making them personalized and grounded in a larger array of
features. This has as noted the potential of improving sales, activity and user
satifisfaction.

% Limited amount of data -> we need to utilize other sources. Implicit!
As the application is not yet officially released there is a limited amount of
user-item interaction from a set of beta users. In addition users does not have
the possibility to explicitly rate items on a numeric scale, a scheme often
employed by machine learning engineers and application developers in order to
understand their users preferences. Combining these two factors makes the
recommendation scenario both interesting and unique, since we need to utilize
the implicit information contained in user behaviour such as clicking, wanting
or buying an item. Limited data both in number of users, but also in activity,
makes the system prone to what is called \textit{cold-start issues}, where
making recommendations with limited information is resolved by a variety of
techniques.

% The fashion domain have certain challenges.
This scenario differs from making recommendations on movies, books and music,
not just in the lack of rich explicit ratings, but also in the context of the
domain the recommendations will be done, namely the fashion domain. Firstly,
fashion consumption is largely determined by seasons. E.g. one does not buy
winter jackets in the middle of summer. Most clothes do also go out of
fashion, the same can not be said for \emph{all} movies. Secondly, there are a
whole different set of important aspects regarding the items or products when
recommending in the fashion domain, such as: brand, color and size. 

% Brand and price are important!
Finally, users are highly price and brand-aware. This can be examplified by
looking at an average movie consumer, where typically the director of the movie
does not gratly affect the way the movie is consumed. However, in the fashion
domain the consumer might mainly look at the product brand when deciding what
to consume. Price preferences are also related to this property where some
users prefer making \textit{a good deal} on periodic sales whilst other buys
clothes not only for induvidual satisfaction, but to show of or make a
statement.

% What are our goals and purpose?
In our system, there are two sources of information available for making
recommendations: an aggregated product database and event logs capturing user
acitivities. Out first goal is to find a technique for translating the implicit
feedback into preferences. These preferences are then combined with product
features and techniques for mitigating the cold-start problem. Finally the
user-item preferences are utilized in making recommendations where challenges
related to both domain and data sources are considered.

% Example scenario
For the users of SoBazaar the final product should not infer with the already
established user interface. Instead the news feed will evolve from today
showing the same recommendations to all users, to showing \textit{personalized
recommendations} to every user, based on their \textit{behaviour} in the
application. For every user-item interaction the user is implicitily changing
his/hers preference-profile and will then, by just using the application, get
novel and more robust recommendations. Further, as personalized recommendations
hopefully will increase both sales and user activity we will, with a larger set
of data, be able to understand more user patterns and consequently gain a
deeper understanding of the domain.

\section{Problem Statement and Goals}

\marginpar{TODO: Overall problem statement}
%Problem statement
The primary aim of this thesis is to propose a design for a recommender system, that may be used in an
e-commerce fashion application. The designed system should be able to generate recommendations based on
user-interactions with the applications, and optionally use product information to improve the recommendation
quality. In addition, the design should provide a solution to the problem of giving accurate recommendations
to new users of the system.

Recommender systems have been applied to a wide range of different domains including music, movies, e-commerce, news
and many others. Since each application domain has its own specific needs, the methods used for recommendations differ.
This leads us to our first research goals, which are the following:

\begin{itemize}
	\item G1: Gain a better understanding of the fashion domain.
  	\item G2: Identify the specific challenges of making fashion recommendations.
  	\item G3: How can existing technologies be adapted to mitigate or overcome these challenges?
\end{itemize}

Most recommender systems base their recommendations of previous feedback given by the user. A central problem for
recommender systems is therefore the cold-start problem. How do you recommend items when you have little or no
user feedback. SoBazaar is a \emph{brand new} application, and have therefore naturally recorded a limited amount
of user-interactions. Poor recommendations can result in customer defection and loss of revenue for Telenor.
The above reasoning lead us to the following goal, which is to:

\begin{itemize}
  \item G4: Find the existing solutions to the cold-start problem presented in the recommender system literature
  		and present the possible solution(s) best suited for our application and needs.
\end{itemize}

We focus mainly on finding \emph{complete} solutions to the cold-start problem, that can handle both new users, new items
and general sparsity related problems.

As previously mentioned, recommender systems base their recommendations on feedback given by the user. SoBazaar records
all the users interactions with the application, which then could be interpreted as user feedback given by users to items.
We have multiple types of interactions such as e.g. browsing, wanting and purchasing items. We would like to figure out
how these events could be used to learn the users true preferences. This lead us to the following goals:

\marginpar{TODO: Assumptions and Constraints}
%E.g. Due to data limitations we have excluded cold-start articles looking at how demographic data can be incorporated to solve the cold-start user problem.

\begin{itemize}
 	\item G5: Explore the existing solutions of how to infer user preference from implicit feedback data.
 	\item G6: Establish user interaction patterns to support our assumptions.
	\item G7: Find different methods of combining various event types into \emph{implicit ratings}?
  	\item G8: Evaluate the \emph{implicit ratings}.
\end{itemize}

We would like to combine the solutions found through our work with the above mentioned goals and combine these in a
proposal to a design for a fashion e-commerce recommender system.

\section{System Overview}

  \marginpar{Include filterbots in a more detailed system overview?}
  
  This thesis proposes a system for making recommendations based on implicit
  data, where the input is implicit feedback collected on a set of users and
  items and the output is personalized recommendations to all users, proposing
  new and relevant products to their preferences.
 
  \begin{figure}[H]
    \centering
    \begin{tikzpicture}
      [node distance = 1cm, auto,font=\footnotesize,
      % STYLES
      every node/.style={node distance=1.5cm},
      % The comment style is used to describe the characteristics of each process
      comment/.style={rectangle, inner sep= 5pt, text width=4cm, node distance=0.25cm, font=\scriptsize\sffamily},
      % The nonProcess style
      nonProcess/.style={rectangle, draw, inner sep=5pt, text width=4cm, text badly centered, minimum height=1.2cm, font=\footnotesize\sffamily},
      % The process style is used to draw the processs' name
      process/.style={rectangle, draw, fill=black!10, inner sep=5pt, text width=4cm, text badly centered, minimum height=1.2cm, font=\bfseries\footnotesize\sffamily}]

      % Draw processs
      \node [nonProcess] (inputData) {Input data based on implicit feedback (events)};
      \node [process, below of=inputData] (implicitConverter) {Convert implicit feedback to implicit ratings};
      \node [nonProcess, below of=implicitConverter] (ratings) {Ratings};
      \node [process, below of=ratings] (recommendations) {Recommending};

      \node [comment, right=0.25 of inputData] (comment-inputData) {
        Implicit Feedback (clicks, purhcases etc.) collected from the SoBazaar
        analytics logs. Detailed description in Section~\ref{sec:sobazaar-data}
      };

      \node [comment, right=0.25 of implicitConverter] (comment-implicitConverter) {
        Converts the inputed implicit feedback to implicit ratings based on
        different conversion schemes. Detailed description in
        Section~\ref{sec:implementation-implicit}
      };

      \node [comment, right=0.25 of ratings] (comment-ratings) {
        Set of rating triplets, on the form \textit{user, product, rating}.
      };

      \node [comment, right=0.25 of recommendations] (comment-recommendations) {
        Makes recommendations based on the inputed ratings. Different
        approaches to make the recommendations can be take, such as matrix
        factorization or neighborhood based approaches. Detailed description in
        Section~\ref{sec:making-recommendations}
      };

      % Draw the links between processs
      \path[->,thick]
        (inputData) edge (implicitConverter)
        (implicitConverter) edge (ratings)
        (ratings) edge (recommendations);

    \end{tikzpicture}
    \caption{Overview of the system. Boxes in white represents input and output
    data. Boxes in gray represents processes}
  \end{figure}

  Traditionally in most recommender systems this process does not include the
  first two stages of our system, hence going directly from a set of explicitly
  provided ratings to a set of recommendations. However, as we will see, making
  ourselves non-dependent on explicit ratings can both provide us with a good
  set of recommendations and take us one step closer to true artificial
  intelligence, as we require no extra effort from the users – rather learning
  preferences from their behaviour.

  \paragraph{The limitiations} of this thesis are primarly linked to having
  only one available dataset with both sparse and modest data. As a result,
  making conclusions based user behaviour proved difficult due to lack of
  statistical significance on many observations. Limitiations in both time
  available and number of existing users with sufficient activity made is also
  impossible to conduct a thorough user study, which could confirm possible
  found user patterns. 

  In addition, as seen in the overview, external sources of information such as
  social networks and trust-based networks are not considered – as the dataset
  provided did not warrant this. As the products in the SoBazaar application
  are aggregated from multiple sources, and hence product databases, the
  quality of information are notably diverse. Consequently, since extracting
  basic content features different from product titles became close to
  impossible without large engineering efforts, the main focus does not lie on
  hybrid recommender systems (combining content and user-based
  recommendations), although some work is done in this area.

\section{Outline}
\begin{table}[H]
  \centering
  \begin{tabular}{lp{11cm}}
  \toprule
    \textbf{Chapter}      & \textbf{Description} \\ 
  \midrule

    Chapter 1 & \textbf{The Introduction} chapter gives an overview of the
    project to the reader. It also outlines the purpose and motivation of the
    project.  \\[1.5ex]

    Chapter 2 & \textbf{The Dataset} chapter presents the results from our
    dataset analysis. \\[1.5ex]

    Chapter 3 & \textbf{The Preliminary Study} chapter documents knowledge,
    research and technology that is relevant to the project, and how and why
    some of them were prioritized over others when it comes to how they are
    used in the project. \\[1.5ex]

    Chapter 4 & \textbf{The implementation} chapter describes the design of the
    system and how the design has be implemented. \\[1.5ex]

    Chapter 5 & \textbf{Evaluation} chapter discussed the development process,
    testing of results and major issues. \\[1.5ex]

    Chapter 6 & \textbf{The Conclusion} chapter sums up the project and describes the
    findings and reflects on them. It also describes further work to be done.
    \\[1.5ex]

    Appendix & \textbf{The Appendix} contains extended information such as a
    full list of the requirements. \\

  \bottomrule
  \end{tabular}
  \caption{Overview of structure and chapters in the thesis}
  \label{table-reportstructure}
\end{table}
