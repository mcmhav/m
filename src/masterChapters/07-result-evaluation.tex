% !TEX root = ../report.tex

\chapter{Evaluation}
\label{chap:resulteval}
\minitoc

%TODO intro

\clearpage

% !TEX root = ../../report.tex
\section{Experimental Results}

In this section we present the experimental results...



\section{Binary Purchase Only Dataset}

\begin{table}
    \centering
    \begin{tabular}{*{5}l}
    \toprule
    Model 			&	AUC			&	$MAP@20$ \\ \midrule
    Most Popular	&	0.30314		&	0.00000	\\
    ItemBasedKNN	&	0.52366		&	0.00000	\\
    UserBasedKNN	&	0.52189		&	0.00000	\\
    BPR-MF			&	0.34490		&	0.00000	\\
    \bottomrule
    \end{tabular}
\caption[Experimental Results - Purchase Only Dataset]{Experimental Results for the purchase only dataset using random splits. The results are averaged over 5 runs}
\end{table}


\section{Binary Multiple Events Dataset}



\begin{table}
    \centering
    \resizebox{\columnwidth}{!}{%
    \begin{tabular}{*{15}l}
    \toprule
    Model 			&	AUC			&	$MAP@20$		&	$P_c$ 	& 	$P_w$ 	& 	$P_p$ 	& 	$A_c$ 	& 	$A_w$ 	& 	$A_p$ 	& 	$R_c	$	&	$R_w$		& 	$R_p$	& 	$MAP@_p$	& 	$MAP@_w$ & 	$MAP@_p$ \\ \midrule
    Most Popular	&	0.00000		&	0.00000		&	0.00000	& 	0.00000	& 	0.00000	& 	0.00000	& 0.00000	& 0.00000	& 	0.00000	& 	0.00000	& 	0.00000 & 	0.00000	& 	0.00000	&   0.00000\\
    %ItemBasedKNN	&	
    %UserBasedKNN	&	
    %BPR-MF			&	
    \bottomrule
    \end{tabular}
    }
\caption{Experimental Results - Purchase Only Dataset}
\end{table}




\subsection{The SoBazaar Dataset (Implicit Ratings)}

This subsection will present the experimental results for the SoBazaar dataset using implicit ratings.


\begin{table}
    \centering
    \resizebox{\columnwidth}{!}{%
    \begin{tabular}{*{15}l}
    \toprule
    Model 			&	AUC			&	$MAP@20$		&	$P_c$ 	& 	$P_w$ 	& 	$P_p$ 	& 	$A_c$ 	& 	$A_w$ 	& 	$A_p$ 	& 	$R_c	$	&	$R_w$		& 	$R_p$	& 	$MAP@_p$	& 	$MAP@_w$ & 	$MAP@_p$ \\ \midrule
    Most Popular	&	0.00000		&	0.00000		&	0.00000	& 	0.00000	& 	0.00000	& 	0.00000	& 0.00000	& 0.00000	& 	0.00000	& 	0.00000	& 	0.00000 & 	0.00000	& 	0.00000	&   0.00000 \\
    %ItemBasedKNN	&	
   % UserBasedKNN	&	
   % ALS-WR			&	
    \bottomrule
    \end{tabular}
    }
\caption{Experimental Results - Linear Count}
\end{table}




\begin{table}
    \centering
    \resizebox{\columnwidth}{!} &
    \multicolumn{1}{c}{60\%} &
    \multicolumn{1}{c}{80\%} &
    \multicolumn{1}{c}{40\%} &
    \multicolumn{1}{c}{60\%} &
    \multicolumn{1}{c}{80\%} &
    \multicolumn{1}{c}{40\%} &
    \multicolumn{1}{c}{60\%} &
    \multicolumn{1}{c}{80\%} &
    \multicolumn{1}{c}{40\%} &
    \multicolumn{1}{c}{60\%} &
    \multicolumn{1}{c}{80\%} &
    \multicolumn{1}{c}{40\%} &
    \multicolumn{1}{c}{60\%} &
    \multicolumn{1}{c}{80\%} &
    \multicolumn{1}{c}{40\%} &
    \multicolumn{1}{c}{60\%} &
    \multicolumn{1}{c}{80\%} \\ \midrule
    MostPopular 		& 0.5982 & 0.6028 & 0.6071 & 0.0254 & 0.0256 & 0.0187 & 0.0136 & 0.0156 & 0.0089 & 5.7803 & 5.7803 & 2.8902 & 1.0  & 1.0  & 1.0  & 1.0 	& 1.0  & 1.0						\\
    MostPopular + FB 	& 0.7557 & 0.7687 & 0.7714 & 0.0254 & 0.0256 & 0.0174 & 0.0157 & 0.0156 & 0.0085 & 5.7803 & 5.7803 & 2.8902 & 1.0  & 1.0  & 1.0  & 1.0 	& 1.0  & 1.0  						\\
    Rec 3				& -		 & - 	  & - 	   & - 		& - 	 & - 	  & -	   & -		& - 	 & -	  & -	   & - 		& -	   & -	  &	-	 & -	& -    & - 							\\

    \bottomrule
    \end{tabular}
    }
    \caption{SoBazaar Cold-start System Evaluation Results using Implicit Ratings (count\_sigmoid\_fixed\_sr-3.5.txt)}
\end{table}

\begin{table}
\centering
\resizebox{\columnwidth}{!}{%
\begin{tabular}{|c|*{18}{c|}c|}
\hline
&	 \multicolumn{3}{c|}{AUC} & \multicolumn{3}{c|}{nDCG@20} &	 \multicolumn{3}{c|}{MAP@20} &	 \multicolumn{3}{c|}{HLU} & \multicolumn{3}{c|}{IS Coverage} & \multicolumn{3}{c|}{US Coverage} \\ \hline
Model 				& 10\% 	 & 40\%   & 75\%   & 10\%   & 40\%   & 75\%   & 10\%   & 40\% 	& 75\%	 & 10\%   & 40\%   & 75\%   & 10\% & 40\% & 75\% & 10\% & 40\% & 75\%   					\\ \hline
MostPopular 		& 0.7385 & 0.7391 & 0.7533 & 0.0247 & 0.0400 & 0.0474 & 0.0087 & 0.0135 & 0.0219 & 5.7143 & 7.4561 & 8.7336 & 1.0  & 1.0  & 1.0  & 1.0 	& 1.0  & 1.0						\\ \hline
MostPopular + FB 	& 0.8569 & 0.8587 & 0.8672 & 0.0254 & 0.0400 & 0.0474 & 0.0089 & 0.0135 & 0.0219 & 2.0179 & 5.7143 & 8.7336 & 1.0  & 1.0  & 1.0  & 1.0 	& 1.0  & 1.0						\\ \hline
Rec 3				& -		 & - 	  & - 	   & - 		& - 	 & - 	  & -	   & -		& - 	 & 	   	  & -	   &		&	   &	  &		 &		&	   & - 							\\ \hline
Rec 4				& -		 & - 	  & - 	   & - 		& - 	 & - 	  & -	   & -		& - 	 & 	   	  & -	   &		&	   &	  &		 &		&	   & - 							\\ \hline
\end{tabular}
}
\caption{SoBazaar Cold-start Item Evaluation Results using Implicit Ratings (count\_sigmoid\_fixed\_sr-3.5.txt)}
\end{table}

\begin{table}
\centering
\resizebox{\columnwidth}{!}{%
\begin{tabular}{|c|*{18}{c|}c|}
\hline
&	 \multicolumn{3}{c|}{AUC} & \multicolumn{3}{c|}{nDCG@20} &	 \multicolumn{3}{c|}{MAP@20} &	 \multicolumn{3}{c|}{HLU} & \multicolumn{3}{c|}{IS Coverage} & \multicolumn{3}{c|}{US Coverage} \\ \hline
Model 				& 10\% 	 & 40\%   & 75\%   & 10\%   & 40\%   & 75\%   & 10\%   & 40\% 	& 75\%	 & 10\%   & 40\%   & 75\%   & 10\% & 40\% & 75\% & 10\% & 40\% & 75\%   					\\ \hline
MostPopular 		& 0.9775 & 0.9807 & 0.9811 & 0.0804 & 0.0937 & 0.1045 & 0.0449 & 0.0468 & 0.0533 & 4.0595 & 6.0879 & 7.6238 & 1.0  & 1.0  & 1.0  & 1.0 	& 1.0  & 1.0						\\ \hline
MostPopular + FB 	& 0.9877 & 0.9895 & 0.9897 & 0.0804 & 0.0937 & 0.1045 & 0.0449 & 0.0468 & 0.0533 & 4.0595 & 6.0879 & 7.6238 & 1.0  & 1.0  & 1.0  & 1.0 	& 1.0  & 1.0						\\ \hline
Rec 3				& -		 & - 	  & - 	   & - 		& - 	 & - 	  & -	   & -		& - 	 & 	   	  & -	   &		&	   &	  &		 &		&	   & - 							\\ \hline
Rec 4				& -		 & - 	  & - 	   & - 		& - 	 & - 	  & -	   & -		& - 	 & 	   	  & -	   &		&	   &	  &		 &		&	   & - 							\\ \hline
\end{tabular}
}
\caption{SoBazaar Cold-start User Evaluation Results using Implicit Ratings (count\_sigmoid\_fixed\_sr-3.5.txt)}
\end{table}


\begin{table}
    \centering
    \resizebox{\columnwidth}{!} &
    \multicolumn{1}{c}{60\%} &
    \multicolumn{1}{c}{80\%} &
    \multicolumn{1}{c}{40\%} &
    \multicolumn{1}{c}{60\%} &
    \multicolumn{1}{c}{80\%} &
    \multicolumn{1}{c}{40\%} &
    \multicolumn{1}{c}{60\%} &
    \multicolumn{1}{c}{80\%} &
    \multicolumn{1}{c}{40\%} &
    \multicolumn{1}{c}{60\%} &
    \multicolumn{1}{c}{80\%} &
    \multicolumn{1}{c}{40\%} &
    \multicolumn{1}{c}{60\%} &
    \multicolumn{1}{c}{80\%} &
    \multicolumn{1}{c}{40\%} &
    \multicolumn{1}{c}{60\%} &
    \multicolumn{1}{c}{80\%} \\ \midrule
NameID: MostPopular  mode: item &   0.9833  &   0.6840  &   0.6203  &   0.0000  &   0.0000  &   0.0000  &   0.0435  &   0.0606  &   0.0647  &   0.0000  &   0.0000  &   0.0000  &   0.9992  &   0.9992  &   0.9992  &   1.0000  &   1.0000  &   1.0000 \\
NameID: MostPopular  mode: system   &   0.5679  &   0.5913  &   0.5916  &   0.0029  &   0.0024  &   0.0015  &   0.0220  &   0.0204  &   0.0201  &   0.0000  &   0.4149  &   0.0000  &   0.9989  &   0.9990  &   0.9991  &   1.0000  &   1.0000  &   1.0000 \\
NameID: MostPopular  mode: user &   0.7664  &   0.7514  &   0.7405  &   0.0000  &   0.0000  &   0.0000  &   0.0081  &   0.0029  &   0.0024  &   0.0000  &   0.0000  &   0.0000  &   0.9992  &   0.9992  &   0.9992  &   1.0000  &   1.0000  &   1.0000 \\
NameID: Random   mode: item &   0.4893  &   0.4979  &   0.4985  &   0.0007  &   0.0015  &   0.0021  &   0.0000  &   0.0004  &   0.0016  &   0.0000  &   0.2301  &   0.0967  &   0.9992  &   0.9992  &   0.9992  &   1.0000  &   1.0000  &   1.0000 \\
NameID: Random   mode: system   &   0.5210  &   0.4918  &   0.5188  &   0.0031  &   0.0040  &   0.0009  &   0.0002  &   0.0023  &   0.0008  &   0.0000  &   0.0000  &   0.0000  &   0.9989  &   0.9990  &   0.9991  &   1.0000  &   1.0000  &   1.0000 \\
NameID: Random   mode: user &   0.5055  &   0.5108  &   0.4934  &   0.0012  &   0.0008  &   0.0005  &   0.0014  &   0.0007  &   0.0001  &   0.0000  &   0.0000  &   0.0000  &   0.9992  &   0.9992  &   0.9992  &   1.0000  &   1.0000  &   1.0000 \\
NameID: Zero     mode: item &   0.0170  &   0.3164  &   0.3794  &   0.0000  &   0.0000  &   0.0000  &   0.0000  &   0.0000  &   0.0000  &   0.0000  &   0.0000  &   0.0000  &   0.0000  &   0.0000  &   0.0000  &   0.0000  &   0.0000  &   0.0000 \\
NameID: Zero     mode: system   &   0.5619  &   0.5504  &   0.5614  &   0.0000  &   0.0045  &   0.0018  &   0.0000  &   0.0010  &   0.0007  &   0.0000  &   0.0000  &   0.0000  &   0.0000  &   0.0000  &   0.0000  &   0.0000  &   0.0000  &   0.0000 \\
NameID: Zero     mode: user &   0.7184  &   0.7115  &   0.6864  &   0.0079  &   0.0031  &   0.0009  &   0.0024  &   0.0008  &   0.0006  &   0.0000  &   0.0000  &   0.0000  &   0.0000  &   0.0000  &   0.0000  &   0.0000  &   0.0000  &   0.0000 \\
NameID: svd  mode: item &   0.8101  &   0.6013  &   0.5748  &   0.0134  &   0.0132  &   0.0158  &   0.0362  &   0.0304  &   0.0275  &   0.6303  &   0.6135  &   1.0155  &   1.0000  &   0.9992  &   1.0000  &   1.0000  &   1.0000  &   1.0000 \\
NameID: svd  mode: system   &   0.5498  &   0.5549  &   0.5699  &   0.0179  &   0.0065  &   0.0044  &   0.0045  &   0.0057  &   0.0050  &   1.3274  &   0.0000  &   0.0000  &   0.9989  &   0.9990  &   1.0000  &   1.0000  &   1.0000  &   1.0000 \\
NameID: svd  mode: user &   0.7814  &   0.7788  &   0.7743  &   0.0013  &   0.0012  &   0.0005  &   0.0189  &   0.0058  &   0.0034  &   0.0000  &   0.0000  &   0.0000  &   0.9992  &   0.9992  &   0.9992  &   1.0000  &   1.0000  &   1.0000 \\
NameID: ItemKNN  mode: item &   0.2779  &   0.4035  &   0.4405  &   0.0000  &   0.0000  &   0.0000  &   0.0014  &   0.0059  &   0.0101  &   0.0000  &   0.0000  &   0.0000  &   0.6422  &   0.5735  &   0.6134  &   0.9174  &   0.9159  &   0.9180 \\
NameID: ItemKNN  mode: system   &   0.5555  &   0.5729  &   0.5729  &   0.0161  &   0.0015  &   0.0057  &   0.0067  &   0.0005  &   0.0011  &   3.0973  &   0.0000  &   0.0000  &   0.8922  &   0.9039  &   0.9455  &   0.9948  &   0.9530  &   0.9355 \\
NameID: ItemKNN  mode: user &   0.7377  &   -1.0000 &   0.7337  &   0.0121  &   -1.0000 &   0.0003  &   0.0000  &   nan &   0.0028  &   0.0000  &   -1.0000 &   0.8000  &   0.8491  &   0.8173  &   0.9508  &   0.9314  &   0.8943  &   0.9337 \\
NameID: WRMF     mode: item &   0.8422  &   0.6315  &   0.5858  &   0.0070  &   0.0102  &   0.0091  &   0.0416  &   0.0358  &   0.0374  &   0.6303  &   0.9202  &   0.9671  &   0.9992  &   0.9992  &   0.9992  &   1.0000  &   1.0000  &   1.0000 \\
NameID: WRMF     mode: system   &   0.5230  &   0.5329  &   0.5592  &   0.0080  &   0.0014  &   0.0052  &   0.0030  &   0.0040  &   0.0032  &   0.8850  &   0.0000  &   0.8000  &   0.9989  &   0.9990  &   0.9991  &   1.0000  &   1.0000  &   1.0000 \\
NameID: WRMF     mode: user &   0.7602  &   0.7580  &   0.7507  &   0.0013  &   0.0026  &   0.0000  &   0.0130  &   0.0044  &   0.0025  &   0.0000  &   0.5333  &   0.0000  &   0.9992  &   0.9992  &   0.9992  &   1.0000  &   1.0000  &   1.0000 \\
NameID: UserKNN  mode: item &   0.9426  &   0.6679  &   0.6094  &   0.0000  &   0.0000  &   0.0000  &   0.0418  &   0.0294  &   0.0280  &   0.0000  &   0.0000  &   0.0000  &   0.9992  &   0.9992  &   0.9992  &   1.0000  &   1.0000  &   1.0000 \\
NameID: UserKNN  mode: system   &   0.5508  &   0.5672  &   0.5684  &   0.0023  &   0.0066  &   0.0014  &   0.0035  &   0.0112  &   0.0123  &   0.0000  &   0.0000  &   0.0000  &   0.9989  &   0.9990  &   0.9991  &   1.0000  &   1.0000  &   1.0000 \\
NameID: UserKNN  mode: user &   0.7925  &   0.7979  &   0.7839  &   0.0016  &   0.0000  &   0.0000  &   0.0184  &   0.0072  &   0.0062  &   0.0000  &   0.0000  &   0.0000  &   0.9992  &   0.9992  &   0.9992  &   1.0000  &   1.0000  &   1.0000 \\
NameID: BPRMF    mode: item &   0.9214  &   0.6576  &   0.6071  &   0.0012  &   0.0000  &   0.0000  &   0.0285  &   0.0352  &   0.0384  &   0.0000  &   0.0000  &   0.0000  &   0.9992  &   0.9992  &   0.9992  &   1.0000  &   1.0000  &   1.0000 \\
NameID: BPRMF    mode: system   &   0.5885  &   0.6359  &   0.6072  &   0.0027  &   0.0025  &   0.0045  &   0.0030  &   0.0043  &   0.0055  &   0.4425  &   0.0000  &   0.4000  &   0.9989  &   0.9990  &   0.9991  &   1.0000  &   1.0000  &   1.0000 \\
NameID: BPRMF    mode: user &   0.7490  &   0.7408  &   0.7361  &   0.0018  &   0.0005  &   0.0007  &   0.0042  &   0.0016  &   0.0014  &   0.0000  &   0.0000  &   0.2667  &   0.9992  &   0.9992  &   0.9992  &   1.0000  &   1.0000  &   1.0000 \\
    \bottomrule
    \end{tabular}
    }
    \caption{Testur}
\end{table}


Explanation:
\begin{table}
    \centering
    \resizebox{\columnwidth}{!}{%
    \begin{tabular}{ll}
    \toprule
    Variable & Description \\
    \midrule
    AUC &  \\
    MAP@20 & Mean average precision  \\
    T\_c & Test  \\
    T\_w &  \\
    T\_p &  \\
    P\_c &  \\
    P\_w &  \\
    P\_p &  \\
    R\_c &  \\
    R\_w &  \\
    R\_p &  \\
    MAP@20-click &  \\
    MAP@20-want &  \\
    MAP@20-purchase &  \\
    \bottomrule
    \end{tabular}
    }
    \caption{Testur}
\end{table}

\newcommand{\Testur}{
\begin{table}\centering\resizebox{\columnwidth}{!}{\begin{tabular}{*{19}l}\toprule
 & AUC &        MAP@20 &        T\_c &  T\_w &  T\_p &  P\_c &  P\_w &  P\_p &  R\_c &  R\_w &  R\_p &  MAP@20-click &  MAP@20-want &   MAP@20-purchase &        \\
\midrule
Count sigmoid   &       -1 &    0 &     1309 &  1327 &  144 &   0 &     0 &     0 &     0 &     0 &     0 &     0 &     0 &     0 &      \\
Recentness sigmoid      &       -1 &    0 &     1323 &  1274 &  132 &   0 &     0 &     0 &     0 &     0 &     0 &     0 &     0 &     0 &      \\
Recentness linear       &       -1 &    0 &     1332 &  1313 &  135 &   0 &     0 &     0 &     0 &     0 &     0 &     0 &     0 &     0 &      \\
Count linear    &       -1 &    0 &     1394 &  1241 &  145 &   0 &     0 &     0 &     0 &     0 &     0 &     0 &     0 &     0 &      \\
\bottomrule\end{tabular}}\caption{svd mahout}\end{table}
\begin{table}\centering\resizebox{\columnwidth}{!}{\begin{tabular}{*{19}l}\toprule
 & AUC &        MAP@20 &        T\_c &  T\_w &  T\_p &  P\_c &  P\_w &  P\_p &  R\_c &  R\_w &  R\_p &  MAP@20-click &  MAP@20-want &   MAP@20-purchase &        \\
\midrule
Count linear    &       -1 &    0.001838 &      1394 &  1241 &  145 &   0 &     0 &     0 &     0 &     0 &     0 &     0.001346 &      0.000297 &      0.001961 &       \\
Recentness sigmoid      &       -1 &    0.001122 &      1323 &  1274 &  132 &   0 &     0 &     0 &     0 &     0 &     0 &     0.002281 &      0.000844 &      0 &      \\
Recentness linear       &       -1 &    0.001517 &      1332 &  1313 &  135 &   0 &     0 &     0 &     0 &     0 &     0 &     0.0025 &        0.00344 &       0.000387 &       \\
Count sigmoid   &       -1 &    0.000565 &      1309 &  1327 &  144 &   0 &     0 &     0 &     0 &     0 &     0 &     0.001334 &      0.00035 &       0.001041 &       \\
\bottomrule\end{tabular}}\caption{loglikelihood mahout}\end{table}
\begin{table}\centering\resizebox{\columnwidth}{!}{\begin{tabular}{*{19}l}\toprule
 & AUC &        MAP@20 &        T\_c &  T\_w &  T\_p &  P\_c &  P\_w &  P\_p &  R\_c &  R\_w &  R\_p &  MAP@20-click &  MAP@20-want &   MAP@20-purchase &        \\
\midrule
Count linear    &       -1 &    0 &     1394 &  1241 &  145 &   121 &   106 &   13 &    0.086801 &      0.085415 &      0.089655 &      0 &     0 &     0 &      \\
Recentness sigmoid      &       -1 &    0.000012 &      1323 &  1274 &  132 &   111 &   108 &   17 &    0.0839 &        0.084772 &      0.128788 &      0 &     0 &     0 &      \\
Recentness linear       &       -1 &    0.000076 &      1332 &  1313 &  135 &   127 &   136 &   14 &    0.095345 &      0.10358 &       0.103704 &      0 &     0 &     0 &      \\
Count sigmoid   &       -1 &    0 &     1309 &  1327 &  144 &   127 &   109 &   16 &    0.097021 &      0.08214 &       0.111111 &      0 &     0.000207 &      0 &      \\
\bottomrule\end{tabular}}\caption{UserKNN None}\end{table}
\begin{table}\centering\resizebox{\columnwidth}{!}{\begin{tabular}{*{19}l}\toprule
 & AUC &        MAP@20 &        T\_c &  T\_w &  T\_p &  P\_c &  P\_w &  P\_p &  R\_c &  R\_w &  R\_p &  MAP@20-click &  MAP@20-want &   MAP@20-purchase &        \\
\midrule
Count linear    &       -1 &    0 &     1394 &  1241 &  145 &   141 &   128 &   17 &    0.101148 &      0.103143 &      0.117241 &      0 &     0 &     0.000207 &       \\
Recentness sigmoid      &       -1 &    0 &     1323 &  1274 &  132 &   124 &   123 &   8 &     0.093726 &      0.096546 &      0.060606 &      0 &     0.000018 &      0 &      \\
Recentness linear       &       -1 &    0.000021 &      1332 &  1313 &  135 &   125 &   116 &   17 &    0.093844 &      0.088347 &      0.125926 &      0 &     0.000011 &      0 &      \\
Count sigmoid   &       -1 &    0 &     1309 &  1327 &  144 &   131 &   124 &   11 &    0.100076 &      0.093444 &      0.076389 &      0 &     0 &     0.000364 &       \\
\bottomrule\end{tabular}}\caption{ItemKNN None}\end{table}
\begin{table}\centering\resizebox{\columnwidth}{!}{\begin{tabular}{*{19}l}\toprule
 & AUC &        MAP@20 &        T\_c &  T\_w &  T\_p &  P\_c &  P\_w &  P\_p &  R\_c &  R\_w &  R\_p &  MAP@20-click &  MAP@20-want &   MAP@20-purchase &        \\
\midrule
Count sigmoid   &       -1 &    0.009564 &      1309 &  1327 &  144 &   106 &   125 &   11 &    0.080978 &      0.094197 &      0.076389 &      0.007667 &      0.005249 &      0.008223 &       \\
Recentness sigmoid      &       -1 &    0.004292 &      1323 &  1274 &  132 &   92 &    123 &   9 &     0.069539 &      0.096546 &      0.068182 &      0.006348 &      0.009645 &      0.017195 &       \\
Recentness linear       &       -1 &    0.006055 &      1332 &  1313 &  135 &   117 &   118 &   13 &    0.087838 &      0.089871 &      0.096296 &      0.004981 &      0.003477 &      0.000813 &       \\
Count linear    &       -1 &    0.006872 &      1394 &  1241 &  145 &   108 &   122 &   18 &    0.077475 &      0.098308 &      0.124138 &      0.003901 &      0.02623 &       0.003874 &       \\
\bottomrule\end{tabular}}\caption{UserKNN item recommender}\end{table}
\begin{table}\centering\resizebox{\columnwidth}{!}{\begin{tabular}{*{19}l}\toprule
 & AUC &        MAP@20 &        T\_c &  T\_w &  T\_p &  P\_c &  P\_w &  P\_p &  R\_c &  R\_w &  R\_p &  MAP@20-click &  MAP@20-want &   MAP@20-purchase &        \\
\midrule
Count linear    &       -1 &    0.015234 &      1394 &  1241 &  145 &   57 &    36 &    5 &     0.04089 &       0.029009 &      0.034483 &      0.01478 &       0.009365 &      0.011881 &       \\
Recentness sigmoid      &       -1 &    0.014354 &      1323 &  1274 &  132 &   63 &    45 &    2 &     0.047619 &      0.035322 &      0.015152 &      0.013854 &      0.011252 &      0.002665 &       \\
Recentness linear       &       -1 &    0.014338 &      1332 &  1313 &  135 &   71 &    35 &    8 &     0.053303 &      0.026657 &      0.059259 &      0.015405 &      0.007984 &      0.022969 &       \\
Count sigmoid   &       -1 &    0.006715 &      1309 &  1327 &  144 &   55 &    49 &    2 &     0.042017 &      0.036925 &      0.013889 &      0.010057 &      0.006602 &      0.001176 &       \\
\bottomrule\end{tabular}}\caption{MostPopular item recommender}\end{table}
\begin{table}\centering\resizebox{\columnwidth}{!}{\begin{tabular}{*{19}l}\toprule
 & AUC &        MAP@20 &        T\_c &  T\_w &  T\_p &  P\_c &  P\_w &  P\_p &  R\_c &  R\_w &  R\_p &  MAP@20-click &  MAP@20-want &   MAP@20-purchase &        \\
\midrule
Count linear    &       -1 &    0.000344 &      1394 &  1241 &  145 &   3 &     3 &     0 &     0.002152 &      0.002417 &      0 &     0.000397 &      0.000529 &      0.000335 &       \\
Recentness sigmoid      &       -1 &    0.000161 &      1323 &  1274 &  132 &   1 &     3 &     0 &     0.000756 &      0.002355 &      0 &     0.000825 &      0.00076 &       0 &      \\
Recentness linear       &       -1 &    0.000107 &      1332 &  1313 &  135 &   8 &     3 &     0 &     0.006006 &      0.002285 &      0 &     0.000076 &      0.000113 &      0 &      \\
Count sigmoid   &       -1 &    0.000386 &      1309 &  1327 &  144 &   6 &     3 &     0 &     0.004584 &      0.002261 &      0 &     0.000501 &      0.000875 &      0.000784 &       \\
\bottomrule\end{tabular}}\caption{Random item recommender}\end{table}
\begin{table}\centering\resizebox{\columnwidth}{!}{\begin{tabular}{*{19}l}\toprule
 & AUC &        MAP@20 &        T\_c &  T\_w &  T\_p &  P\_c &  P\_w &  P\_p &  R\_c &  R\_w &  R\_p &  MAP@20-click &  MAP@20-want &   MAP@20-purchase &        \\
\midrule
Count sigmoid   &       -1 &    0.000555 &      1309 &  1327 &  144 &   8 &     19 &    2 &     0.006112 &      0.014318 &      0.013889 &      0.000901 &      0.000043 &      0 &      \\
Recentness sigmoid      &       -1 &    0.000139 &      1323 &  1274 &  132 &   13 &    23 &    2 &     0.009826 &      0.018053 &      0.015152 &      0.000216 &      0.001771 &      0.000415 &       \\
Recentness linear       &       -1 &    0.000049 &      1332 &  1313 &  135 &   15 &    18 &    1 &     0.011261 &      0.013709 &      0.007407 &      0.000723 &      0.000072 &      0 &      \\
Count linear    &       -1 &    0.000139 &      1394 &  1241 &  145 &   15 &    21 &    0 &     0.01076 &       0.016922 &      0 &     0.000193 &      0.002742 &      0.000861 &       \\
\bottomrule\end{tabular}}\caption{ItemKNN item recommender}\end{table}

}

\Testur

\subsection{Does our proposed implicit rating methods improve the recommendation quality over binary preference data?}

Does our findings support our hypothesis?

\subsection{Compare the different implicit rating functions}

Does our findings support our hypothesis?

\subsection{Select the best combination of methods for the SoBazar recommender system}

Does our findings support our hypothesis?






\section{Discussion}

In Chapter 1 we introduced a total of 8 goals for our Master's thesis:

\begin{itemize}
\item G1: Gain a better understanding of the fashion domain.
\item G2: Identify the specific challenges of making fashion recommendations.
\item G3: Study how existing technologies can be adapted to mitigate or
  		  overcome these challenges.
\item G4: Study existing solutions to the cold-start.
\item G5: Identify the best suited methods with regard to both application and domain.
\item G6: Explore the existing solutions of how to infer user preference from implicit feedback data.
\item G7: Establish user interaction patterns to support our assumptions.
\item G8: Find different methods of combining various event types into implicit ratings.
\item G9: Find metrics in order to evaluate the \emph{implicit ratings}
\end{itemize}

In the following subsection we will reiterate these goals and discuss whether we have succeeded in reaching our goals. Readers should
note that we already have performed a detailed analysis of the results of the work on goals G1 in Section \ref{sec:dataset-conclusion},
G2 in Section \ref{}, G3 in Section \ref{}, G4 and G5 in Section \ref{sec:cold-start-discussion}, G6 in Section \ref{implicit-weaknesses},
G7 in Section \ref{}, G8 in Section \ref{} and finally G9 in Section \ref{sec:evaluation-metrics}.
The following discussion will therefore be fairly high-level, and avoid the \emph{nitty-gritty} details.

\subsection{G1, G2 and G3: Solutions to the fashion domain related problems}
\label{sec:fashion-discussion}

Through thorough examination and analysis of the data and a literature review of fashion articles we came to the following main conclusions...

%Important notes 

\subsection{G4: The Cold-Start Problem}
\label{sec:cold-start-discussion}

Through our literature review, we closely examined five different classes of solutions to the cold-start problem. Namely trust-aware recommender system, filterbots, seed users, interview process and hybrid methods.

However, a question that should be asked is whether there exists and solutions which we did not discover or was excluded from our review. An issue that could have influenced the results of our review is \emph{researchers bias}, which could have arisen of any of the researchers had preferences for one of the solution types before the literature search. The reason for including multiple methods that never were used was due to uncertainty regarding what data would become available during the project. Trust-aware recommender system and hybrid methods in particular could never be tested out due to a lack of social/demographic data, which we were hoping to get access to.



After closely examining the different methods we concluded that both filterbots and content-boosted hybrid methods could provide a possible solution for
our specific scenario...

Demographic information could further improve its performance and usefulness. Agarwal et. al. \cite{Agarwal2009} used 13 filterbots in their experiments, where 11 out of 13 bots rated items based on demographic information.


\subsection{G5, G6, G7: Implicit Ratings}
\label{sec:implicit-discussion}

Importance of implicit factors for the domain

Another question that arises is: when does it become obsolete to look at clicks and wants?







\section{Issues}\label{sec:issues}

%TODO - Mention that the recommender system libraries are terribad for evaluation...
