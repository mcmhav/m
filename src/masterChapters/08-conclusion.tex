% !TEX root = ../report.tex

\chapter{Conclusion}
\minitoc

\clearpage

\section{Final Product}


\section{Related Work}


\section{Future Work}



As previously mentioned in the evaluation section the main reasons for implementing a recommender
system is the desire to improve user satisfaction and to increase the economic success of a platform.
We believe it would be interesting to look into how one could implement a reward attribute for the
items, that factors in how much Telenor will profit from its sale.

\begin{equation}
ExpectedReturn_i = P(Sale_i) * Reward(i)
\end{equation}

The question then, is how this information can e.g. be used/incorporated into the recommender to
increase profits without sacrificing (too much) user satisfaction.


%Discussion on product database features to improve content-based recommendations
%Missing color, category, age-group and others
However, the content-based filtering methods require rich descriptions of items and well built 
and well informed user profiles. These ideal cases are rare in real applications. This dependence 
on the quality and structure of data is the main weakness of methods based on content.


%Recommender systems for anonymous users - Context
	%Recommend other items with a high similarity to the item currently being viewed

%Social recommendations (Trust)

%Demographic filtering as a solution to the cold-start user problem?
	
% Personalization of rating generators	
%	 I guess the min and max scores can be estimated for each users if we have enough data.
% 	 Then we can personalize the scores and provide personalized recommendations. 


