% !TEX root = ../report.tex

\chapter{Conclusion}
\minitoc

In this thesis we have first...

Presented the SoBazaar application and the sobazar dataset.

State of the art 


\clearpage

\section{Final Product}


\section{Related Work}


\section{Future Work}



As previously mentioned in \ref{sec:experimental-plan} sec the main reasons for implementing a recommender
system is the desire to improve user satisfaction and to increase the economic success of a platform.
We believe it would be interesting to look into how one could implement a reward attribute for the
items, that factors in how much Telenor will profit from its sale.

\begin{equation}
ExpectedReturn_i = P(Sale_i) * Reward(i)
\end{equation}

The question then, is how this information can e.g. be used/incorporated into the recommender to
increase profits without sacrificing (too much or any) user satisfaction by recommending
more \emph{expensive} items to the user.


%Discussion on product database features to improve content-based recommendations
%Missing color, category, age-group and others
However, the content-based filtering methods require rich descriptions of items and well built 
and well informed user profiles. These ideal cases are rare in real applications. This dependence 
on the quality and structure of data is the main weakness of methods based on content.


With regards to recommendation quality we believe the main focus should be on getting more
and higher quality data.

\begin{itemize}
\item More extensive crawling the webstores collecting likes or popularity information, better content descriptions,
	  and so on.	 
\item Provide the user with \emph{cool} and \emph{fun} ways of expressing their preferences. A tinder-like \emph{hot or not}
	  interface, the ability to follow brand or friends are a few such examples.
\item Negative feedback in the form of a dislike button or similar.
\end{itemize}

%Recommender systems for anonymous users - Context
	%Recommend other items with a high similarity to the item currently being viewed

Contextual item-to-item recommendations should be considered implemented to display related
items to anonymous users instead of the current system which recommends "People how love this also love".


%Demographic filtering as a solution to the cold-start user problem?
	
%Finding and incorporating more implicit factors
	
%Personalization of rating generators	
%	 I guess the min and max scores can be estimated for each users if we have enough data.
% 	 Then we can personalize the scores and provide personalized recommendations. 
%Personalized implicit factors -> Connection to SVD and true big-data methods
%which basicly boils every user and item down to a set of factors and utility
%values for these factors.

As shown in \cite{FranceTelecom} most successful commercial recommender incorporate multiple
sources of information. The following information sources should therefore be examined more
closely to assess if they can add any value to the system:

\begin{itemize}
\item Social information. Trust could be inferred from functionality such as \emph{follow user}, which
	  could be used to improve recommendation quality and improve cold-start performance.
\item Content information. 
\item Demographic information.
\end{itemize}

\marginpar{Short discussion on the above mentioned information sources and how they can help. BE EXPLICIT!}
%Importance of brand
%Importance of social groups - fashion



This thesis have only on a small part of the possibilities. There is much work to be done. %%% THE END %%%
	
