% !TEX root = ../report.tex

\chapter{Conclusion}

In this thesis we have first presented our analysis of the Sobazaar dataset, a highly sparse dataset comprising 
multiple event types which could be considered as implicit feedback. Our main findings (what?) from the data analysis
lead us to do a state-of-the art review of cold-start solutions, the fashion domain and methods for converting
implicit feedback into implicit ratings.

Our state-of-the-art and related work section focused in particular on the cold-start problem where we evaluated the 
different solution types with respect to initial quality of service or cold-start performance and user-effort required before the system can provide high quality recommendations.
As a part of the literature review, the solution types were also evaluated in the context of a recommender
system applied to the fashion domain.

%We found that an interview process in the form of a simple hot or not interface
%could be a great solution to the cold-start problem by making it \emph{fun} for users to provide both positive
%and negative feedback, increasing both the quality and amount of user feedback available.



The key findings from the review of the fashion domain ... are important factors to consider when designing a
recommender system for the fashion domain.


and how implicit feedback could be mapped into \emph{implicit ratings}.

The key findings from the literature review and data analysis were then utilized when we presented our
design for different implicit rating mapping functions and how these could be combined.


The proposed implicit rating functions was implemented and tested against recommender systems utilizing binary
preferences only. Our findings indicate that... 







\clearpage




\section{Future Work}




As previously mentioned in \ref{sec:experimental-plan} sec the main reasons for implementing a recommender
system is the desire to improve user satisfaction and to increase the economic success of a platform.
We believe it would be interesting to look into how one could implement a reward attribute for the
items, that factors in how much the retailer will profit from its sale.

\begin{equation}
ExpectedReturn_i = P(Sale_i) * Reward(i)
\end{equation}

The question then, is how this information can e.g. be used/incorporated into the recommender to
increase profits without sacrificing (too much or any) user satisfaction by recommending
more \emph{expensive} items to the user.


%Discussion on product database features to improve content-based recommendations
%Missing color, category, age-group and others
However, the content-based filtering methods require rich descriptions of items and well built 
and well informed user profiles. These ideal cases are rare in real applications. This dependence 
on the quality and structure of data is the main weakness of methods based on content.


With regards to recommendation quality we believe the main focus should be on getting more
and higher quality data.

\begin{itemize}
\item More extensive crawling the webstores collecting likes or popularity information, better content descriptions,
	  and so on.	 
\item Provide the user with \emph{cool} and \emph{fun} ways of expressing their preferences. A tinder-like \emph{hot or not}
	  interface, the ability to follow brand or friends are a few such examples.
\item Negative feedback in the form of a dislike button or similar.
\end{itemize}

Adding data entry amounts by adding a step to your funnel could hurt sales more than it helps. One should
therefore carefully consider adding features the require user effort. If a \emph{hot or not} interface is to be implemented
one should preferably conduct a study to find the best strategy of selecting items to display to the users.

Especially better quality and structure of the data. Table \ref{table:extracted-content-features} shows an overview of
the percentage of items we managed to extract the different attributes for. Attributes such as category, color, brand-name ...
are particulary useful and should be easily extracted from the product database for most products. 

%Recommender systems for anonymous users - Context
	%Recommend other items with a high similarity to the item currently being viewed

Contextual item-to-item recommendations should be considered implemented to display related
items to anonymous users in the context of viewing an item instead of the current system which
recommends "People how love this also love".



	
Finding and incorporating more implicit factors should also be considered as a possible future direction.

Do trends repeat year after year?

Personalization of the impact of different implicit factors could be interesting to look at.
Implicit factors such as brand preference, importance of global popularity, weight given
to the different event types and so on could all be personalized for each user.
However, we feel it is important to mention the connection to matrix factorization methods
aka SVD which \emph{basicly} does the same thing automatically. It reduced the rating matrix
to a set of item and users factors for each individual item and user and a set of utility values
for these factors.


Other implicit factors such as brand preference, ... could all be implemented to
	


As shown in \cite{FranceTelecom} most successful commercial recommender incorporate multiple
sources of information. The following information sources should therefore be examined more
closely to assess if they can add any value to the system:

\begin{itemize}
\item Social information. Trust could be inferred from functionality such as \emph{follow user}, which
	  could be used to improve recommendation quality and improve cold-start performance.
\item Content information. 
\item Demographic information.
\end{itemize}

\marginpar{Short discussion on the above mentioned information sources and how they can help. BE EXPLICIT!}
%Importance of brand
%Importance of social groups - fashion
%Demographic filtering as a solution to the cold-start user problem?


This thesis have only on a small part of the possibilities. There is much work to be done. %%% THE END %%%
	
