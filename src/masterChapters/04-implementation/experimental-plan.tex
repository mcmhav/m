% !TEX root = ../../report.tex
\clearpage
\section{Experimental Plan}
\label{sec:experimental-plan}

%We can sometimes evaluate how well the recommender achieves its overall goals.
%For example, we can check an e-commerce website revenue with and without the
%recommender system and thereby estimate the value of the system to the website.

In this section we present our plan to validate our methods. We first introduce
the main goals for our experiment, then discuss how we plan to achieve them.
The second part therefore touches on our evaluation methodology.
We have the following main goals for our experiment:

\begin{itemize}
	\item Goal 1: determine the effect of utilizing multiple event types
	\item Goal 2: determine whether our proposed implicit rating methods improve the recommendation quality over
		  		  using \emph{more simplistic} binary methods.
	\item Goal 3: evaluate the different implicit factors and attempt to quantify their importance.
	\item Goal 4: quantify if our cold-start solution adds any value to the system.
	\item Goal 5: propose a \emph{best} combination of methods for the SoBazar recommender system.
\end{itemize}

The following subsections will describe our methodology for testing and verifying these goals:

\subsection{Utilizing multiple event types}

We want to measure if there is any added value to using multiple implicit factors over looking
at purchase data only. Meaning that we want to measure whether a recommender that utilizes
data including multiple implicit factors improve the recommendation quality. This can not
easily be measured using traditional measures due to the fact that the datasets contains
different users and items. In addition to measure their quality by means of different
evaluation metrics one must also make a few assumptions regarding the importance
of the different events. Our main assumptions is that it is better to recommend a clicked
item than a random item, and that the same also applies to wants. The additional coverage
will also be factored in. This means that our comparison will be made on a somewhat
subjective basis.

\subsection{Improved Recommendation Quality}

We want a way to measure if our proposed models outperform the \emph{simpler} one-class
collaborative filtering methods. The main problem with this is that our main contribution
is the ratings that are fed into the recommender, and that we are forced to use different
methods for providing recommendation. Optimally, we would like every variable to be fixed
to get the most reliable results. In an attempt to \emph{mitigate} this problem, we 
have carefully selected methods which we consider to be the state-of-the-art from both segments
to make the comparison as fair as possible. The comparison of recommendation quality
is mainly a evaluation metric problem and will be discussed at greater length in
Section \ref{sec:eval-metrics}.


\subsection{Quantifying the Importance of Implicit factors}

We want to learn which implicit factors are the most \emph{important} with regard to
improved recommendation quality. Our success then relies on that we successfully can compare
a set of ratings, which is something you not often encounter in the literature. As we currently
are limited to offline studies we have to make a few assumptions to execute the comparison.
Our main assumption is that better ratings will produce better recommendation results,
which we believe to be a fair assumption, as ALS-WR especially is designed to take
in ratings with different confidence levels attached to them. Thus, making this
problem mainly an evaluation metric problem, which is discussed at greater
length in Section \ref{sec:eval-metrics}. This allows us to input different sets
of ratings generated incorporating different implicit factors and see which implicit
factors improve our results the most.

\subsection{Cold-start}

We wish to learn whether or not our filterbots enhance our systems ability to handle
the three different cold-start related problems. This means that we need some methodology
to be able to simulate the different cold-start scenarios and measure the recommendation quality.

As mentioned in Section~\ref{sec:cold-start-eval} there is no common framework for assessing 
the cold-start performance of recommender systems. To recap the cold-start problem and how
the dataset changes over time we have three main input types:

\begin{itemize}
	\item 	Existing users watch new items in the catalogue
	\item	New users join the system and view their first item
	\item	New items are added to the catalogue
\end{itemize}

The first input source has the effect of increasing the dataset density, while
the latter two has the opposite effect. To simulate a new user entering the system
we adopt a methodology similar to the one described in \cite{Stern2009, Lam2008}. The
users are split in two disjoint sets, setting aside 10\% of the users as test users.
We introduce a selection criteria similarly to the one described in \cite{Rashid2002, Rashid2008}
where we only select users that have given above 20 ratings as eligible test users,
then selecting a random subset of these users.
We then train three different models using 10\%, 40\% and 75\% of the test users
ratings and try to predict their remaining ratings. The same methodolgy is applied
to test items, the only difference being that only 5\% of the items are used as 
test items due to a low number of average ratings per item. The selection criteria
for items is set to a minimum of 15 ratings.

To evaluate the cold-start system performance we use the same method as
described in ~\cite{Agarwal2009}. We use a 80:20 split where the same 20\% is set
aside for testing. We then randomly draw 40\%, 60\% and finally 80\% (all) ratings
and see how well our method perform for the different sparsity levels.
It is important to note that this process should be repeated multiple times, as
the chance of getting an \emph{unfortunate} split is highly probable due to the
dataset size.

We perform both random and time-based splits. For our time based
splits we train the model with ratings in chronological order, e.g. using
the 10\% of the first ratings for a user, where his later activity is put in the testset.
Similarly for the cold-start system splits we train the model using different random subset
of the 80\% earliest ratings and try to predict the remaining 20\%.

\subsection{Proposing a solution}

Based on the experimental results from \textbf{Goal 1}, \textbf{Goal 2}, \textbf{Goal 3}, and \textbf{Goal 4}
we will propose a solution for the SoBazaar recommender system.

\subsection{Generalization Properties}

In addition to evaluate the methods on the SoBazaar dataset we want to verify that our
methods generalizes beyond our experimental dataset, in accordance to the general guidelines
for experimental studies \cite{Shani2011}. The data used for offline evaluation should match
as closely as possible the data we expect the recommender system to face when it is
deployed \cite{Gunawardana2009}. When selecting datasets for evaluation we focused on the
following dataset properties:

\begin{itemize}
	\item Size of dataset: Preferable as close as possible to SoBazaar or larger (a few months from now)
	in terms of number of ratings, users and items.
	\item Different different types of implicit factors	such as clicks, likes and purchases.
	\item Domain: Preferably a domain as close to possible as the e-commerce domain where
	factors like recency also would apply.
	\item Timestamps: To evaluate the recentness mapping
	\item Presence of features (Secondary)
\end{itemize}

We were unable to acquire any e-commerce datasets containing user browsing history including different event types
and timestamps. Thus making the main contributions of this article untestable, thus rendering further experiments
on other datasets useless. It is worth mentioning that we inquired other researchers having experimented with similar
datasets to no luck.

\paragraph{Experimental Datasets}

As previously mentioned the SoBazaar dataset is both small and sparse.
Having such a small and sparse dataset has several implications. Firstly we have
to avoid \emph{wishful thinking} as we have very thin data, meaning that we cannot
rely on getting reliable results. Secondly, our evaluation methodology must be
\emph{tailored} for small sparse datasets. E.g. when using cross-validation the number
of folds depends on the size of the dataset. For large datasets, even 3-fold Cross
validation will be quite accurate, while for very sparse datasets, one may have to
use leave-one-out in order to train on as many examples as possible. The advantages
of using a large number of folds is that the bias of the true error rate estimators
will be small, meaning that the estimator will be very accurate, with the disadvantages
being that the variance of the true error rate estimator will be large in addition to
increased computation time. Another alternative well suited for sparse datasets is the
\emph{all but one} or the \emph{leave one out} method, in which we remove one rating
from the test users and try to predict the hidden rating.

Another important concern is whether or not to take the timestamps into consideration,
which directly speaks against the use of cross-validation, as we wish to use the past
interactions to predict future actions. When using the \emph{leave one out} method one
could always remove the users freshest rating and try to predict it and repeat the
process any number of times. This is particularly relevant as some of our implicit
mapping functions factors in recency.

We chose to utilize both time-splits and random splits to improve our confidence
in our results. For both splits we repeat the process 10 times to get as reliable
estimates as possible. The following table shows an overview of the experimental datasets.

\begin{table}[H]
    \centering
    \begin{tabular}{l l l l l l }
    \toprule
	Dataset						& 	Ratings		& 	Users		& 	Items 		& 	Sparsity	& Rating Scale 				    \\ \midrule
	Sobazar	(Purchases Only) 	&	1,592		&	466			&	1188		&	99.71243	& Binary						\\
	Sobazar (All events)		& 	27,873  	& 	1,532		&	5688		& 	99.69657	& Binary/Implicit Ratings		\\
	\bottomrule
    \end{tabular}
    \caption [Experimental datasets]{General overview of the datasets used for evaluation}
    \label{table:datasets}
\end{table}

As you can see the recommender can cover a much larger portion of the user group
 and items when including multiple event types. However,
the sparsity is only reduced slightly as the number of items and users are much higher.

