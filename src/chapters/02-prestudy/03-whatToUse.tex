% !TEX root = ../../report.tex

\section{What to use}

%Having a large amount of data like e.g. in the netflix dataset
% -> Do not require a great understanding of the data to get decent results
% -> Out case is a little different. What implications does the limited amount of data have?

%We need to take this into account when designing our system / selecting methods for recomendations


\subsection{Some Awesome Algorithms (Build up with project progress)}

Given enough data, item-based CF methods often performs as well or better than
almost any other recommendation method. However, in cold-start situations where
a user, an item, or the entire system is new, simple non-personalized
recommendations often fare better...

User based - new user
Non personalized approaches
    - most popular
    - highest rated
Other alternatives
    - use demographic information

Item based - new item
    - most popular
    - highest rated
    - use content information

When you enter a clothing store you are normally confronted with the following suggestions:
    - New in/Seasonal highlights
    - Special offer/discounts
    - Bestsellers
    - Are you looking for something in particular?

Personalized recommendations, what assumptions can be made?
\#1 - You are like your friends
\#2 - You are like people who do similar things that you do
\#3 - You like things that are similar to things you already like
\#4 - You are influenced by experts and the opinions of others

\subsubsection{The Good}
\subsubsection{The Bad}
\subsection{Why Not To Use These (Same As above)}
\subsubsection{The Good}
\subsubsection{The Bad}
