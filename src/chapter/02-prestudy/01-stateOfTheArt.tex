% !TEX root = ../../report.tex

\section{State Of The Art}

The following section will present the state of the art recommender systems...

\subsection{System Coldstart Handling}

%When does the system go from a cold->warm?
%When can a user be considered warm? (How many ratings?)
%When can a item be considered warm? (How many ratings?)

In the literature, the term cold is used about an object in a system, or a whole system, which is new. Cold-start scenarios in recommender systems are situations in which little/no prior events, like ratings or clicks, are known for certain users or items. The cold-start problem can therefore be divided into three sub problems:

\begin{itemize}
	\item \emph{Cold-start system}: A situation where we only have new users and little or no ratings for the items.
	\item \emph{Cold-start item}: The problem of recommending items that are new to the system, which have not received any ratings. E.g. in a scenario where the average item in an item collection have 10 000 ratings, a new item with only 5 ratings would be considered an "cold-item"
	\item \emph{Cold-start user}: The problem of giving accurate recommendations who is new to a recommender system. E.g. in a recommender system where the average user have rated 100 items, a user which only have rated 2 items, would be considered a "cold-user".
\end{itemize}

The cold-start system problem is mainly a collaborative filtering problem, and can be seen as a combination of the cold-start user and cold-start item problem where the majority of the users are new to the system and have expressed few preferences, resulting in a very sparse user-item matrix, rendering recommendation systems using traditional collaborative-filtering methods futile.

The cold-start user problem is present both in content-based and collaborative-filtering systems. In content-based systems, the lack of ratings given by the target user, means that the target user will have a limited content-profile, since the users content profile is constructed using content-information from his/hers rated items. Similarly, the cold-start user problem also affects user-based collaborative filtering, since user similarities are calculated based on items that a user shares with other users of the system. The system will therefore most likely struggle to find users with tastes that are \emph{truly} similar to the target user. In both cases, recommendation quality is most likely bound to suffer.

%TODO: Cite articles using content-information to tackle the cold start item problem
The cold-start item problem. In content-based systems, new items can easily be recommended using the content information of the item, making it a popular solution \cite{•} to the cold-start new item problem. This problem is more severe in collaborative-filtering systems where items are only recommendable if they have been rated by multiple users. New items will therefore not be recommendable before multiple users somehow stumble upon the new item while e.g. browsing the item collection, unless additional measures are taken to solve this problem.

\subsubsection{Cold-start system}

In this section we will present a few different solutions to the cold-start system problem found in the literature. As mentioned in the introduction to the cold-start problem, most algorithms only work effectively in environments where the dataset has high information density. In fact, in extreme cases, when data is very scarce, simple non-personalized recommendations based on global averages can outperform collaborative-filtering and content-based algorithms. Pure collaborative filtering cannot help in a cold-start setting, since no user preference information is available to form any basis for recommendations.

%Propose a categorization of approaches
	%Initial Categorization
		%The immediate remedy - Meta information (Hybrid approaches)
		%Blending/Non personalized approaches/Global models for all users
		%Seed items or users
			%Wisdom of the Better Few: Cold Start Recommendation via Representative based Rating Elicitation
		%Trust-Aware Collaborative Filtering
			%Trust-Aware Collaborative Filtering for Recommender Systems
			%+ many others
		%Filterbots / surrogate users
			%Na¨ıve Filterbots for Robust Cold-Start Recommendations
				%Check out RipperBots (For binary rating version)
				%Recommend items based on the average of the users with the same demographics...

%Read a few articles that fall under each category and present them
%Discussion: Which are the most relevant?
	%How do we determine what is the best solution?

\subsubsection{Cold-start user}

This section will present a few different solutions to the cold-start user problem proposed in the literature.

%Can the different approaches be classified? E.g. 3 main categories of approaches
	%Initial categorization
		%Interview process to quickly learn user preference/Ask to rate, Intelligent selection
		As pointed out by Rashid et. al. \cite{Rashid2002}, the most direct way of acquiring information for use in personalised recommendations from a new users is to present item for the user to rate...
			%Information theory vs most popular, How do one select the items to present to the user?
				%Rashid2002, Rashid2008
 			%Decision trees
 				%Functional Matrix Factorizations for Cold-Start Recommendation
 		%Hybrid approaches
 			%Using demographic data - KEEP IN MIND: What kind of demographic information is most commonly used? / Most informative in our case?
 			It is a common strategy to gather demographic data in order to gain knowledge about the user. Data that may be collected typically includes age, gender, nationality, marital status, income, educational level and occupation. The idea is that people with a more common background share a more similar taste than someone with a random background, and therefore good recommendations can be made as long as we know the new user’s background.
 		%Key figures / Seed users. NB! For later, check with Seed items or users (Cold-start system)
 			%Whom Should I Trust? The Impact of Key Figures on Cold Start Recommendations
 		%%Trust-aware / Trust propagation

%Read a few articles from each category
%Discussion: Which category/approach is the most relevant for us?
	%How do we determine this "systematically"?

\begin{quotation}
Ask the right questions if you're going to find the right answers
\end{quotation}
- Vanessa Redgrave


Scenario: The target user has very few ratings (e.g. a new user). In this scenario, collaborative filtering (CF) based recommenders might not be able to find users with tastes that are truly similar to the target user, thus the recommendation quality to the target user might be poor. On the other hand, because of the very limited number of items rated by the target user, it is hard to obtain the content interests of the target user. Consequently, content-based techniques might only generate very limited recommendations in such situations.

One crucial problem of recommender system is how to best learn from new users. Collaborative Filtering (CF), is the best known technology for recommender systems and is based on the idea that like-minded users have similar tastes and preferences. A new user therefore poses a challenge to CF recommender, since the system has no knowledge about the preferences of the new user, and can therefore not provide any personalized recommendations, this is known as the cold start problem for new users. The system must therefore acquire some information about the new user in order to make personalized recommendations.

However, the system must be careful to present useful items to garner information. A food recommender should probably not ask whether a new user likes vanilla ice cream since most people like vanilla ice cream. Therefore, knowing that a new user likes vanilla ice cream tells you very little about the user. The choice of what questions to ask a new user, then, is critical.

Rashid et. al. \cite{Rashid2002} performed a study of different item selection strategies that collaborative filtering recommender systems can use to learn about new users. They presented the users with a questionnaire with items asking them to rate/select the ones they like. Their strategies can be divided into five classes:

\begin{itemize}
\item \emph{Random:} strategies: Strategies that avoid bias in the presentation of bias
\item \emph{Popularity:} Select among the top N items where the probability that is proportionate to the items popularity.
\item \emph{Pure entropy:} Present the items with the highest entropy that the user has not seen
\item \emph{Balanced strategies:} A balanced approach combining both popularity data and entropy.
\item \emph{Personalized:} As soon as some information is known about a user, present items specifically tailored to that user using e.g. item-item similarity
\end{itemize}

This study was later extended by Rashid et. al. \cite{Rashid2008} where they more closely examined information theoretic strategies for item selection.

%Their suggestion for e-commerce: Recommend most popular items rather than the highest rated ones, and then use item-item similarity as quickly as possible

A new user preference elicitation strategy needs to ensure that the user does not 1) lose interest in returning due to low quality initial recommendations, 2) as quickly as possible being able to provide good personalized recommendations (find the right neighborhoods).

We are constrained to unobtrusively learn user-profiles from the natural interactions of users with the system, meaning that we can not require the user to rate e.g. 10 items before we can start providing recommendations. We have a \emph{mixed initiative} system meaning that there is provisions for both user and system controlled interactions. We (the system) can only select which items to recommend to the user, and this does not mean that the user actually will click an item or rate it.


\subsubsection{Cold-start item}

To solve the new-item problem, there are two commonly used (simple) solutions that often are used in E-commerce websites:

\begin{itemize}
\item Advertising at the homepage/front-page of the website, putting the new items in an eye catching position. This solution, however, may 		result in that some users, which don't like these new items, might leave the website.
\item Requesting the user to choose one or more of his/hers categories while registering for the site, and recommend items from the selected categories. This approach however, requires active user involvement and complicates the sign up process -> BAD, in addition, many users might chose not to give up any personal interest information, thus the user group covered by this solution is not large enough.
\end{itemize}

%What strategies exist? Categorization
	%Most dominant! Hybrid approach incorporating item information
	%Dynamic Browsing tree
		%Collaborative Filtering Cold-Start Recommendation Based on Dynamic Browsing Tree Model in E-commerce


%Read a few articles for each category, summarize them
%What is suitable in our case?/Discussion

%TODO: Categorize these articles
% Regression-based Latent Factor Models - http://dl.acm.org/citation.cfm?id=1557029
% Learning Attribute-to-Feature Mappings for Cold-Start Recommendations - http://ieeexplore.ieee.org/xpl/abstractCitations.jsp?tp=&arnumber=5693971&url=http%3A%2F%2Fieeexplore.ieee.org%2Fxpls%2Fabs_all.jsp%3Farnumber%3D5693971
% fLDA: Matrix Factorization through Latent Dirichlet Allocation - http://dl.acm.org/citation.cfm?id=1718499
% Matchbox: Large Scale Bayesian Recommendations

\subsection{Fashion Recommendation}

% Building Recommender Systems using a Knowledge Base of product semantics
% http://images.accenture.ca/SiteCollectionDocuments/PDF/recommenderws02.pdf
% 	- Would probably require some more product semantics...

%What are the challanges of making recommendations for fashion?

%	- For how long are items relevant?
	%	- Spring, Summer, Fall, Winter collections
			%Improving E-Commerce Recommender Systems by the Identification of Seasonal Products (Article)
	%	- Freshness, fresh decay operators

%	- Implicit feedback (Based around users fashion browsing habits and an occational purchase...)
	%	- What do we look at? What information is the most useful
	%		- Item category, item keywords, brand... ?

%	- Changing interest of users
%	- Unstructured content/multiple content providers
	% - How to select features for a content-based approach
		% E.g. keywords, when descriptions are in multiple languages
%	- Sparsity
	% - Can rating infromation from similar items be used to decrease sparsity? (Content infromation - Hybrid approaches)
%	- Trends?
	%	- How important is e.g. item popularity?

\subsection{Session Based Recommendation}
Init Hypothesis:
Two users with similar session habits and similar product accessing pattern
have a stronger correlation to one-another than two users with just similar
product interests.


'product\_purchase\_intended' (user pushed to the product web store) shows a
wider specter of information about the product, including additional colors,
images and colors.  For some it might be natural to explore the item there
before "wanting" it. Making both

"product\_purchase\_intended" $\Rightarrow$ "product\_wanted"

and

"product\_purchase\_intended" $\notimplies$ "product\_wanted"

produce valuable information.

Must make different rules for the different stores:
"Bik Bok", "Cubus", "Gina Trik", "H\&M", "Bianco" has a broad specter of extra
functions inside the web store, whereas others might not, only shows the
product and a add to chart button.  This might divide the use pattern of the
users into a:

"product\_detail\_clicked" $\Rightarrow$ "product\_purchase\_intended" $\Rightarrow$ "product\_wanted"

"product\_detail\_clicked" $\Rightarrow$ "product\_purchase\_intended" $\notimplies$ "product\_wanted",

and

"product\_detail\_clicked" $\Rightarrow$ "product\_wanted"

based on the store accessed.

Use this to make a "rule set" with a probability.
Then again use this to recommend items for the users with that given
probability.

Find a "most popular session"-pattern
Find a "most likely to come after"-pattern

% db.sessions.group({key:{'storefront_name':1},cond:{},reduce:function(cur,result){result.count += 1}, initial: {count:0}})

Articles 4 l8er:
%http://dl.acm.org/citation.cfm?id=1136004
%http://link.springer.com/chapter/10.1007/3-540-46119-1_42
%http://dl.acm.org/citation.cfm?id=1082567
%http://link.springer.com/chapter/10.1007%2F978-3-540-30214-8_20
%http://dl.acm.org/citation.cfm?id=502935
%http://dl.acm.org/citation.cfm?id=1835896
%http://dl.acm.org/citation.cfm?id=345169
%http://dl.acm.org/citation.cfm?id=345169

Session issues:
Once in a blue moon a user will do a "product action" (purchase,want,details)
without having a previous frontstore-access event. Which leads to unknown
store-id of the item.

Issue is most probably from missing user-id in collection\_viewed, and a user
checks out an item from there. It is not possible to be 100\% sure which user
access the item from the collection\_viewed event, so this event is therefor
not integrated into the session-stack.


% thoughts:
Categorize stores
    prize
    items in store

Categorize items
    Type
    Prize
    View frequency

Predicting events...
    Value brought vs. clustering on the "item"-events value

Make a store

% event_id, events:

% Products:
%     "product_detail_clicked",
%     "product_wanted",
%     "wantlist_menu_entry_clicked"
%     "product_purchase_intended",

% Store clicked: (produces not NULL storefront_name) (db.sessions.find({'storefront_name':{$ne:'NULL'},$or:[{'event_id':'featured_storefront_clicked'},{event_id:'storefront_clicked'}]}).count())
%     "storefront_clicked",
%     "featured_storefront_clicked",

% Other store interactions
%     "store_clicked",
%     "around_me_clicked",
%     "stores_map_clicked",
%     "collection_viewed",
%     "featured_collection_clicked",

% Start:
%     "app_first_started",
%     "app_became_active",
%     "app_started",
%     "user_logged_in",
%     "facebook_login_failed",

% Other:
%     "friend_invited",
%     "activity_clicked",
%     "facebook_share_changed",

% Course:
%     App started
%     Check next events, a days timeframe


%Simple session form, no structure:
% {u'event_id': u'product_detail_clicked', u'count': 68.0}
% {u'event_id': u'product_wanted', u'count': 35.0}
% {u'event_id': u'storefront_clicked', u'count': 69.0}
% {u'event_id': u'app_started', u'count': 26.0}
% {u'event_id': u'featured_storefront_clicked', u'count': 4.0}
% {u'event_id': u'user_logged_in', u'count': 9.0}
% {u'event_id': u'product_purchase_intended', u'count': 2.0}
% {u'event_id': u'around_me_clicked', u'count': 7.0}
% {u'event_id': u'stores_map_clicked', u'count': 1.0}
% {u'event_id': u'store_clicked', u'count': 1.0}
% {'user_id': 100001385800886L}
% {'num_events': 222}
% Total amount:    222
% User:            100001385800886
% Total Sessions:  30
% Total Events:    936
% Date:            11 - 10 - 2013

% Structured session exploration: Probably more info in this
% > db.sessions.find({'user_id':1094505588,session:64},{'event_id':1,'server_time_stamp':1,'_id':0}).sort({'ts':1})

% > db.sessions.find({'user_id':100000140823565,session:440},{'event_id':1,'server_time_stamp':1,'_id':0}).sort({'ts':1})

% > db.sessions.find({'user_id':100000140823565,session:440},{'product_id':1,'event_id':1,'_id':0}).sort({'ts':1})



\subsection{Recommenders (Similar systems? somethingsomething)}
\subsection{Items clustering}




