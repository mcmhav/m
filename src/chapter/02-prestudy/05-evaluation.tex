% !TEX root = ../../report.tex

\section{How to evaluate}

%What will be adressed in the section?

	%Implicit feedback vs. explicit feedback
		%How is implicit feedback related to user preference
		
	%Evaluation of implicit factors in e-commerce
		% How to model interest in items? Normalization etc...
		% Evaluating Various Implicit Factors in E-commerce (Article)
	
	%What evaluation methodologies exist?
		%Online vs Offline
		%User experiments
		%Questionaires...
		
	%What evaluation measures/metrics exist
		%What is the difference between metrics for explicit and implicit feedback?
		%Discussion on relevance


Some key questions in evaluating recommender systems on testbed data are: what to predict, how to grade performance and what baseline to compare with.


Explicit feedback ("relevance judgement")

Explicit feedback are more precise than implicit feedback, but more difficult
to collect since it requires the user to spend time rating items and the amount
of feedback is often scarce. The main difference between the two is that
implicit feedback is inferred from user behaviour, such as noting which news
articles they do and do not view. Explicit feedback unlike implicit feedback
provides the users with a mechanism to unequivocally express their ratings on a
scale from usually in the form of a Likert scale ranging from 1-5 (strongly
disagree - strongly agree). Thus explicit feedback captures both positive and
negative feedback, while implicit feedback only can be positive. Furthermore,
explicit feedback tend to concentrate on either side of the rating scale, as
users are more likely to express their preference if they feel strongly for or
against an item.

product\_wanted, in the form of like/no feedback, can this be considered
"explicit feedback?" no negative feedback...

Implicit feedback (Indirectly reflect opinion through observing user behaviour)

Unlike the more extensively researched explicit feedback, we do not have any
direct input from the user regarding their personal preferences. In particular
we do not have any substantial evidence of which items the user dislikes (e.g. a
low rating for a movie). But in return implicit feedback is more easily
collected, and usually more abundant.

Almost all of the research on implicit feedback has considered how behaviors can
be used as positive evidence, rather than negative evidence. However, one can imagine
behaviors which indicate that a user does not find something relevant or which suggest
that something is unimportant to the user, such as delete. It is likely that little research
has been conducted on negative implicit feedback because there are fewer of these types
of behaviors, and, in general, less is understood about how to effectively use negative
feedback, whether for implicit or explicit relevance feedback.

Another challenge facing implicit feedback research is the notion of degree of
personalization offered by the system. In particular, individual differences can greatly
impact the effectiveness of using behavior as implicit relevance feedback. People behave
differently and have varying approaches to information-seeking; thus, it is difficult to
generate, and dangerous to apply, all-purpose rules for describing how behavior can be
used as implicit relevance feedback

%Implicit feedback classfication (Type, confidence, precision)

In the case of our project we collect the following data, which can be
considered as implicit feedback.  - product\_detail\_clicked,
product\_purchase\_intended, collection\_viewed(?)...

Other types of implicit feedback include purchase history, browsing
history, search pattern and even mouse movements

Hu et. al. \cite{Hu2008} identify four unique characteristics of implicit
feedback, which differentiates it from explicit feedback...

\begin{enumerate}

\item No negative feedback. By observing user behaviour we can infer which
items the user consume and probably like. However, it is hard to infer which
items the user did not like. This asymmetry has several implications; Explicit
feedback provides a more detailed picture of the
users preferences, but for implicit data the low ratings are treated as missing
data and omitted from the analysis. Hence it is crucial to address the missing
data where most negative feedback is expected to be found

\item Implicit feedback is inherently noisy. While we track user behaviour, we
can only guess their preferences and true motives. For example, a purchase does
not necessarily indicate a positive view of an item, the item may have been
purchased as a gift, or perhaps the user was disappointed
with the item

\item The numerical value of explicit feedback indicates preference, whereas
the numerical value of implicit feedback indicates confidence. Explicit
feedback could e.g. range from total dislike to really like, on the other hand
implicit feedback describe the frequency of actions, e.g. how
frequently a user buys an item. But a higher frequency might not necessarily
indicate a stronger preference. A user might choose to only watch a really good
movie once. However, a recurring event is more likely to reflect the user
opinion. However, the numerical value of the feedback is definitely useful, as
it tells us about the confidence we have in a certain observation

\item Evaluation of implicit feedback requires appropriate measures. In the
case of explicit feedback where a user specify a numerical score, measures such
as mean squared error (MSE) could measure the success of the predictions.
However, with implicit models we have to take into account
the availability of the item, competition with other items, and repeat
feedback.

\end{enumerate}

Accuracy metrics not suited for implicit feedback datasets, as they require
knowing which items are undesired by a user \cite{Hu2008}.

Being accurate is not always enough. Striking a balance between accuracy and
user satisfaction \cite{McNee2006}
  % High accuracy != User satisfaction
  % Therefore, important to consider other evaluation metrics beyond the
  % conventional ones Which are the most important factors to consider with
  % regards to user satisfaction

The data available strongly influences the choice of evaluation method/metrics.
E.g. classification accuracy metrics seem to be the most suitable when working
with binary preferences, e.g. in the form of recall-oriented measures.

\subsection{What Has Been Done Before}

%Evaluation by looking at sessions
%Evaluation using implicit feedback datasets
%Turnover rate?
% ++

%Clues
% http://delivery.acm.org/10.1145/570000/564421/p253-schein.pdf?ip=129.241.103.83&id=564421&acc=ACTIVE%20SERVICE&key=CDADA77FFDD8BE08%2E5386D6A7D247483C%2E4D4702B0C3E38B35%2E4D4702B0C3E38B35&CFID=419807217&CFTOKEN=62708098&__acm__=1394537427_86c608d0d7733db023faa5a09da46de7

\subsection{What To Use}
\subsubsection{The Good}
\subsubsection{The Bad}
