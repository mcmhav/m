% !TEX root = ../report.tex

\chapter{Introduction}
\minitoc
\setcounter{page}{1}
\pagenumbering{arabic}

\clearpage


\section{Purpose}

\section{Motivation}


%What are recommender systems good for?
In today's day and age the increasing amount of data overwhelm our human processing capabilities in many information seeking tasks. To cope with this overload researchers have introduced recommender system to filter the ever increasing information and only present a small selection of items which reflects the users tastes, interests and priorities.  Recommender systems are an active research field and has been successfully applied to many different systems ranging from e-commerce sites such as \emph{Amazon}, movie and TV-series streaming services like \emph{Netflix} and in different music applications such as {Last.fm} and {iTunes}. 

%What is telenors incentive?
Many/most/all of the largest commerce Web sites have been using recommender systems to help their customers find products to purchase for nearly two decades. 
Schafer et. al. \cite{Schafer1999} identified three ways, in which recommender systems increase E-commerce sales: (1) Browser into buyers: Recommender systems can help customers find products they wish to purchase, (2) Cross-sell: Recommender systems improve cross-sell by suggesting additional products for the customer to purchase and (3) Loyalty: In a world where the competitor only is one click away, gaining customer loyalty is an essential customer strategy. Recommender systems improve loyalty by creating a value added relationship between the site and the customer. 

% What are we planning to do? \ Which issues will be investigated?
% Building a recommender system for the app SoBazar (female fashion)
% Pragmatic approach - start off simple, extend system
% Early beta - little data available (Cold start related problems)
	%Social data, log dumps, limited content information...
% Fashion domain (What are the challanges/most important features?)
% How do we evaluate the system?
% Scalability / extensibility issues (Ensure that the system can handle the increased load as it goes live)


\section{Context}
\begin{table}
\centering
\begin{tabularx}{\textwidth}{ l X l }
  \textbf{Chapter}      & \textbf{Description} \\
  \hline \\ [-1.5ex]
  Chapter 1 & The Introduction chapter gives an overview of the project to the reader. It also outlines the purpose and motivation of the project. \\
  \hline \\ [-1.5ex]
  Chapter 2 & The Preliminary Study chapter documents knowledge, research and technology that is relevant to the project, and how and why some of them were prioritized over others when it comes to how they are used in the project. \\
  \hline \\ [-1.5ex]
  Chapter 3 & The Requirements chapter describes the requirements of the project. It also describes how and why they were created. \\
  \hline \\ [-1.5ex]
  Chapter 4 & The Design chapter describes the design of the system and how it was made. \\
  \hline \\ [-1.5ex]
  Chapter 5 & The Implementation chapter describes the implementation of the system. \\
  \hline \\ [-1.5ex]
  Chapter 6 & Evaluation chapter discussed the development process, testing of results and major issues. \\
  \hline \\ [-1.5ex]
  Chapter 7 & The Conclusion chapter sums up the project and describes the findings and reflects on them. It also describes further work to be done. \\
  \hline \\ [-1.5ex]
  Appendix & The appendix contains extended information such as a full list of the requirements. \\
\end{tabularx}
\caption{Structure and chapters of the report.}
\label{table-reportstructure}
\end{table}
